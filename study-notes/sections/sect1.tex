\section{Further Topics on Binomial Option Pricing Model}
\label{sect:binom-model-further}
\begin{enumerate}
\item We have learnt the basics of binomial option pricing model in STAT3905.
Here, we will investigate it in more details and study some further topics
about it.
\end{enumerate}
\subsection{Dynamic Hedging}
\begin{enumerate}
\item In STAT3905, the main tool for pricing an option in a multi-period
binomial tree is \emph{risk-neutral pricing}. Even in the backward induction
process, each step we are utilizing one-period risk-neutral pricing. Can we use
the approach of \emph{pricing by replication}, just like the one-period
binomial tree?

\item It turns out that constructing a replicating portfolio in the
multi-period binomial tree setting is more tricky, since such replicating
portfolio is not ``static'' but ``dynamic'' --- we need to adjust its
components depending on the market movements at the intermediate nodes. This
way of construction is known as \defn{dynamic hedging}.

\begin{center}
\begin{tikzpicture}
\node[] (s0) at (0,0) {\(S_0\)};
\node[] (su) at (4,1) {\(S_u\)};
\node[] (sd) at (4,-1) {\(S_d\)};
\node[] (suu) at (8,2) {\(S_{uu}\)};
\node[] (sud) at (8,0) {\(S_{ud}\)};
\node[] (sdd) at (8,-2) {\(S_{dd}\)};
\draw[-Latex, blue] (s0.east) -- (su.west);
\draw[-Latex, blue] (s0.east) -- (sd.west);
\draw[-Latex, violet] (su.east) -- (suu.west);
\draw[-Latex, violet] (su.east) -- (sud.west);
\draw[-Latex, orange] (sd.east) -- (sud.west);
\draw[-Latex, orange] (sd.east) -- (sdd.west);
\node[] () at (0,3) {Time 0};
\node[] () at (4,3) {Time \(h\)};
\node[] () at (8,3) {Time \(2h\)};
\end{tikzpicture}
\end{center}

\item Recall the backward induction process, which looks like the following.
\begin{center}
\begin{tikzpicture}
\node[] (s0) at (0,0) {\(S_0\)};
\node[] (su) at (4,1) {\(S_u\)};
\node[] (sd) at (4,-1) {\(S_d\)};
\node[] (suu) at (8,2) {\(S_{uu}\)};
\node[] (sud) at (8,0) {\(S_{ud}\)};
\node[] (sdd) at (8,-2) {\(S_{dd}\)};
\draw[-Latex, blue, opacity=0.2] (s0.east) -- (su.west);
\draw[-Latex, blue, opacity=0.2] (s0.east) -- (sd.west);
\draw[-Latex, violet] (su.east) -- (suu.west);
\draw[-Latex, violet] (su.east) -- (sud.west);
\draw[-Latex, orange, opacity=0.2] (sd.east) -- (sud.west);
\draw[-Latex, orange, opacity=0.2] (sd.east) -- (sdd.west);
\node[] () at (0,3) {Time 0};
\node[] () at (4,3) {Time \(h\)};
\node[] () at (8,3) {Time \(2h\)};

\node[below = 1pt of suu, ForestGreen] (vuu) {\(V_{uu}\)};
\node[below = 1pt of sud, ForestGreen] (vud) {\(V_{ud}\)};
\node[below = 1pt of sdd, ForestGreen] (vdd) {\(V_{dd}\)};
\node[below = 1pt of su, ForestGreen] (vu) {\(V_{u}\)\faIcon{check}};
%\node[below = 1pt of sd, brown] {\(V_{d}\)};

\draw[-Latex, dashed, brown] (vuu.west) to[bend right] (vu.north east);
\draw[-Latex, dashed, brown] (vud.west) to[bend left] (vu.south east);
\node[font=\huge, brown] () at (6,0.5) {\faIcon[regular]{arrow-alt-circle-left}};
\end{tikzpicture}

\begin{tikzpicture}
\node[] (s0) at (0,0) {\(S_0\)};
\node[] (su) at (4,1) {\(S_u\)};
\node[] (sd) at (4,-1) {\(S_d\)};
\node[] (suu) at (8,2) {\(S_{uu}\)};
\node[] (sud) at (8,0) {\(S_{ud}\)};
\node[] (sdd) at (8,-2) {\(S_{dd}\)};
\draw[-Latex, blue, opacity=0.2] (s0.east) -- (su.west);
\draw[-Latex, blue, opacity=0.2] (s0.east) -- (sd.west);
\draw[-Latex, violet, opacity=0.2] (su.east) -- (suu.west);
\draw[-Latex, violet, opacity=0.2] (su.east) -- (sud.west);
\draw[-Latex, orange] (sd.east) -- (sud.west);
\draw[-Latex, orange] (sd.east) -- (sdd.west);
\node[] () at (0,3) {Time 0};
\node[] () at (4,3) {Time \(h\)};
\node[] () at (8,3) {Time \(2h\)};

\node[below = 1pt of suu, ForestGreen] (vuu) {\(V_{uu}\)};
\node[below = 1pt of sud, ForestGreen] (vud) {\(V_{ud}\)};
\node[below = 1pt of sdd, ForestGreen] (vdd) {\(V_{dd}\)};
\node[below = 1pt of su, ForestGreen] (vu) {\(V_{u}\)};
\node[below = 1pt of sd, ForestGreen] (vd) {\(V_{d}\)\faIcon{check}};
%\node[below = 1pt of sd, brown] {\(V_{d}\)};

\draw[-Latex, dashed, brown] (vdd.west) to[bend left] (vd.south east);
\draw[-Latex, dashed, brown] (vud.west) to[bend right] (vd.north east);
\node[font=\huge, brown] () at (6,-1.5) {\faIcon[regular]{arrow-alt-circle-left}};
\end{tikzpicture}

\begin{tikzpicture}
\node[] (s0) at (0,0) {\(S_0\)};
\node[] (su) at (4,1) {\(S_u\)};
\node[] (sd) at (4,-1) {\(S_d\)};
\node[] (suu) at (8,2) {\(S_{uu}\)};
\node[] (sud) at (8,0) {\(S_{ud}\)};
\node[] (sdd) at (8,-2) {\(S_{dd}\)};
\draw[-Latex, blue] (s0.east) -- (su.west);
\draw[-Latex, blue] (s0.east) -- (sd.west);
\draw[-Latex, violet, opacity=0.2] (su.east) -- (suu.west);
\draw[-Latex, violet, opacity=0.2] (su.east) -- (sud.west);
\draw[-Latex, orange, opacity=0.2] (sd.east) -- (sud.west);
\draw[-Latex, orange, opacity=0.2] (sd.east) -- (sdd.west);
\node[] () at (0,3) {Time 0};
\node[] () at (4,3) {Time \(h\)};
\node[] () at (8,3) {Time \(2h\)};

\node[below = 1pt of suu, ForestGreen] (vuu) {\(V_{uu}\)};
\node[below = 1pt of sud, ForestGreen] (vud) {\(V_{ud}\)};
\node[below = 1pt of sdd, ForestGreen] (vdd) {\(V_{dd}\)};
\node[below = 1pt of su, ForestGreen] (vu) {\(V_{u}\)};
\node[below = 1pt of sd, ForestGreen] (vd) {\(V_{d}\)};
\node[below = 1pt of s0, ForestGreen] (v0) {\(V_{0}\)\faIcon{check}};

\draw[-Latex, dashed, brown] (vd.west) to[bend left] (v0.south east);
\draw[-Latex, dashed, brown] (vu.west) to[bend right] (v0.north east);
\node[font=\huge, brown] () at (2,-0.5) {\faIcon[regular]{arrow-alt-circle-left}};
\end{tikzpicture}
\end{center}

\item Here, in each step of the backward induction, we shall use the
replicating portfolio approach instead of using the risk-neutral pricing
formula. Through this, we can know how to construct a \emph{dynamic}
replicating portfolio, and thus also know how to \emph{exploit arbitrage
opportunities}, which is the main application of this dynamic hedging process.

\item Let us start at the \(u\) node:
\begin{center}
\begin{tikzpicture}
\node[] (s0) at (0,0) {\(S_0\)};
\node[] (su) at (4,1) {\(S_u\)};
\node[] (sd) at (4,-1) {\(S_d\)};
\node[] (suu) at (8,2) {\(S_{uu}\)};
\node[] (sud) at (8,0) {\(S_{ud}\)};
\node[] (sdd) at (8,-2) {\(S_{dd}\)};
\draw[-Latex, blue, opacity=0.2] (s0.east) -- (su.west);
\draw[-Latex, blue, opacity=0.2] (s0.east) -- (sd.west);
\draw[-Latex, violet] (su.east) -- (suu.west);
\draw[-Latex, violet] (su.east) -- (sud.west);
\draw[-Latex, orange, opacity=0.2] (sd.east) -- (sud.west);
\draw[-Latex, orange, opacity=0.2] (sd.east) -- (sdd.west);
\node[] () at (0,3) {Time 0};
\node[] () at (4,3) {Time \(h\)};
\node[] () at (8,3) {Time \(2h\)};

\node[below = 1pt of suu, ForestGreen] (vuu) {\(V_{uu}\)};
\node[below = 1pt of sud, ForestGreen] (vud) {\(V_{ud}\)};
\node[below = 1pt of sdd, ForestGreen] (vdd) {\(V_{dd}\)};
\node[below = 1pt of su, ForestGreen] (vu) {\(V_{u}\)\faIcon{check}};
%\node[below = 1pt of sd, brown] {\(V_{d}\)};

\draw[-Latex, dashed, brown] (vuu.west) to[bend right] (vu.north east);
\draw[-Latex, dashed, brown] (vud.west) to[bend left] (vu.south east);
\node[font=\huge, brown] () at (6,0.5) {\faIcon[regular]{arrow-alt-circle-left}};
\end{tikzpicture}
\end{center}
We can use the standard formulas for finding out the replicating portfolio
\emph{at \(u\) node}:
\begin{align*}
\Delta_u&=e^{-\delta h}\cdot \frac{V_{uu}-V_{ud}}{S_{uu}-S_{ud}},\\
B_u&=e^{-rh}\cdot \frac{uV_{ud}-dV_{uu}}{u-d}.
\end{align*}
From this we know that the replicating portfolio at \(u\) node should be
comprised of \(\Delta_{u}\) shares of stock and \(B_{u}\) in risk-free bond.
Then of course we can carry out \emph{pricing by replication} and conclude that
\[
V_{u}=\Delta_{u}S_u+B_{u}.
\]
Here, we add a subscript \(u\) to signify that this replicating portfolio is
for \(u\) node.

At the same time point \(h\), if we are at \(d\) node instead of \(u\) node,
the components of replicating portfolio would be different.
\begin{center}
\begin{tikzpicture}
\node[] (s0) at (0,0) {\(S_0\)};
\node[] (su) at (4,1) {\(S_u\)};
\node[] (sd) at (4,-1) {\(S_d\)};
\node[] (suu) at (8,2) {\(S_{uu}\)};
\node[] (sud) at (8,0) {\(S_{ud}\)};
\node[] (sdd) at (8,-2) {\(S_{dd}\)};
\draw[-Latex, blue, opacity=0.2] (s0.east) -- (su.west);
\draw[-Latex, blue, opacity=0.2] (s0.east) -- (sd.west);
\draw[-Latex, violet, opacity=0.2] (su.east) -- (suu.west);
\draw[-Latex, violet, opacity=0.2] (su.east) -- (sud.west);
\draw[-Latex, orange] (sd.east) -- (sud.west);
\draw[-Latex, orange] (sd.east) -- (sdd.west);
\node[] () at (0,3) {Time 0};
\node[] () at (4,3) {Time \(h\)};
\node[] () at (8,3) {Time \(2h\)};

\node[below = 1pt of suu, ForestGreen] (vuu) {\(V_{uu}\)};
\node[below = 1pt of sud, ForestGreen] (vud) {\(V_{ud}\)};
\node[below = 1pt of sdd, ForestGreen] (vdd) {\(V_{dd}\)};
\node[below = 1pt of su, ForestGreen] (vu) {\(V_{u}\)};
\node[below = 1pt of sd, ForestGreen] (vd) {\(V_{d}\)\faIcon{check}};
%\node[below = 1pt of sd, brown] {\(V_{d}\)};

\draw[-Latex, dashed, brown] (vdd.west) to[bend left] (vd.south east);
\draw[-Latex, dashed, brown] (vud.west) to[bend right] (vd.north east);
\node[font=\huge, brown] () at (6,-1.5) {\faIcon[regular]{arrow-alt-circle-left}};
\end{tikzpicture}
\end{center}
In this case, applying the standard formulas again, we would have
\begin{align*}
\Delta_d&=e^{-\delta h}\cdot \frac{V_{ud}-V_{dd}}{S_{ud}-S_{dd}},\\
B_d&=e^{-rh}\cdot \frac{uV_{dd}-dV_{ud}}{u-d}.
\end{align*}
As the input values in the formulas differ, the resulting replicating portfolio
is generally different from that for \(u\) node. Nonetheless, we can again use
pricing by replication to calculate \(V_d\):
\[
V_d=\Delta_{d}S_d+B_d.
\]

Having the values of \(V_u\) and \(V_d\), we can go to the initial node (time
\(0\)) and construct a replicating portfolio there, using the standard formulas
again:
\begin{align*}
\Delta_0&=e^{-\delta h}\cdot \frac{V_{u}-V_{d}}{S_{u}-S_{d}},\\
B_0&=e^{-rh}\cdot \frac{uV_{d}-dV_{u}}{u-d}.
\end{align*}
Of course one can then calculate the time-0 value \(V_0\). But more
importantly, this dynamic hedging approach gives us \emph{three} replicating
portfolios to be constructed:
\[
(\Delta_0,B_0), (\Delta_u,B_u), (\Delta_d,B_d).
\]
How should we actually replicate the values/payoffs of the derivative using
these three pairs of values?

\item To actually perform the replication, we start with the replicating
portfolio at the initial node: \((\Delta_0,B_0)\). By construction, when we
have \(\Delta_0\) shares of stock and \(B_0\) in risk-free bond at time 0, the
portfolio value would match with the value of the derivative for both \(u\) and
\(d\) nodes, i.e., the portfolio value would be \(V_u\) at \(u\) node and
\(V_d\) at \(d\) node. There is nothing special for this period.

However, when we are at time \(h\), the above argument suggests that when we
are at \(u\) or \(d\) node, the replicating portfolio should be
\((\Delta_u,B_u)\) or \((\Delta_d,B_d)\) respectively. Generally,
\emph{adjustments} are needed to \emph{rebalance} our existing portfolio
(resulting from the time-0 replicating portfolio) into one of these, depending
on which node we are at.

More explicitly, the portfolio we have at time \(h\) before rebalancing is
given by \((\Delta_0e^{\delta h},Be^{rh})\) due to the reinvestment of
dividends and accumulation of interest.
\begin{itemize}
\item At \(u\) node, we need to change \((\Delta_0e^{\delta h},Be^{rh})\) to \((\Delta_u,B_u)\).
\item At \(d\) node, we need to change \((\Delta_0e^{\delta h},Be^{rh})\) to \((\Delta_d,B_d)\).
\end{itemize}
To change the components, we need to buy/sell suitable shares of stocks and
borrow/lend suitable amount of money. A natural question then arises: Do these
transactions incur any cost/yield any gain?

\item It turns out that these transactions must be costless, overall speaking. By construction of
the initial replicating portfolio, we must have:
\[
\begin{cases}
\Delta_0e^{\delta h}S_u+Be^{rh}=V_u,\\
\Delta_0e^{\delta h}S_d+Be^{rh}=V_d.\\
\end{cases}
\]
But on the other hand, pricing by replication at \(u\)
and \(d\) nodes suggests that
\[
\begin{cases}
\Delta_u S_u+B_u=V_u,\\
\Delta_d S_d+B_d=V_d.\\
\end{cases}
\]
Combining these two systems, this just means
\[
\begin{cases}
\Delta_0e^{\delta h}S_u+Be^{rh}=\Delta_u S_u+B_u,\\
\Delta_0e^{\delta h}S_d+Be^{rh}=\Delta_d S_d+B_d,\\
\end{cases}
\]
i.e., the total portfolio value \emph{remains unchanged} after the rebalancing!
This explains why those transactions must be overall costless.  Due to this
feature, sometimes the replicating portfolio here is said to be
\defn{self-financing} as it can \emph{finance itself}.

\item After performing the rebalancing at time \(h\), we can just hold the new
replicating portfolio until time \(2h\), to match with the payoff of the
derivative at time \(2h\).

To summarize, in dynamic hedging, generally we need to perform rebalancing at
intermediate nodes, so that we can replicate the payoff of derivative at
\emph{every node}. After obtaining the ``dynamic'' replicating portfolio, we
can use the usual buy-low-sell-high approach to exploit arbitrage opportunity
(if exists). \begin{note}
In this case, we need to be dynamic in the buy-low-sell-high approach. We may
need to perform rebalancing at some intermediate time.
\end{note}
\end{enumerate}

\subsection{Asian Options}
\label{subsect:asian-option}
\begin{enumerate}
\item In STAT3905, we have only considered option which is
\emph{path-independent} in the sense that its payoff depends only the
underlying asset price at the time of expiration (for European option) or time
of exercise (for American option). Here, using the binomial option pricing
model, we will analyze \emph{path-dependent} options whose payoff depends also
on intermediate underlying asset prices (``path''). We will consider two
examples of path-dependent options: \emph{Asian option} (in
\Cref{subsect:asian-option}) and \emph{barrier option} (in
\Cref{subsect:barrier-option}).

\item \defn{Asian option} is a path-dependent option whose payoff depends on a
suitably defined \emph{average price} of the underlying asset over the life of
the option.

To be more specific, we start with a plain vanilla European call with strike
price \(K\).\footnote{Here we focus on \emph{European} Asian option rather than
\emph{American} Asian option.} Its payoff at time \(T\) (maturity) is
\[
(S_T-K)_{+}.
\]
After replacing either \(S_T\) or \(K\) by the ``average price'', we can
incorporate path dependency in the payoff calculation.

Here we have two ways to compute the ``average price'' based on the ``path''
of the underlying asset prices. Given \(n=T/h\) asset prices
\(S_h,S_{2h},\dotsc,S_{nh}\) in \(h\)-year interval, we have:
\begin{enumerate}
\item \defn{arithmetic average}:
\[
A_T=\frac{1}{n}\sum_{i=1}^{n}S_{ih}.
\]
\item \defn{geometric average}:
\[
G_T=\qty(\prod_{i=1}^{n}S_{ih})^{1/n}.
\]
\end{enumerate}
The arithmetic average is more familiar to us and is the ``usual'' way of
computing average. However, it turns out that using geometric average makes the
option pricing more mathematically tractable, especially in the continuous-time
\emph{Black-Scholes model} (in \Cref{sect:bs-model}).

\item After fixing a method for computing average, say geometric average, we
can use \(G_T\) to replace either \(S_T\) or \(K\), and then the resulting
payoff would respectively be
\[
(G_T-K)_{+}\qqtext{or}(S_T-G_T)_{+}.
\]
We get an Asian option in either case.

In the former case, we call the resulting Asian option as \emph{\vc{geometric
average} \blc{price} Asian \orc{call} option}:
\begin{itemize}
\item \vc{geometric average}: the way of computing average
\item \blc{price}: the asset price \(S_T\) gets replaced
\item \orc{call}: we start with a call option
\end{itemize}
There are 3 ``dimensions''.

By the same logic, in the latter case, we call the resulting Asian option as
\emph{geometric average strike Asian call option} (strike \faIcon{arrow-right}
strike price \(K\) gets replaced).

\item In a similar manner, we can get the payoff formulas for the other 6 kinds
of Asian options (there are in total \(2\times 2\times 2=8\) kinds of Asian
options). The payoff formulas for all 8 kinds of Asian option are summarized
below.
\begin{enumerate}[label={(\arabic*)}]
\item arithmetic average price call: \((A_T-K)_{+}\)
\item arithmetic average price put: \((K-A_T)_{+}\)
\item arithmetic average strike call: \((S_T-A_T)_{+}\)
\item arithmetic average strike put: \((A_T-S_T)_{+}\)
\item geometric average price call: \((G_T-K)_{+}\)
\item geometric average price put: \((K-G_T)_{+}\)
\item geometric average strike call: \((S_T-G_T)_{+}\)
\item geometric average strike put: \((G_T-S_T)_{+}\)
\end{enumerate}
\item After knowing the payoff structure of an Asian option, we start
discussing how to \emph{price} an Asian option using the binomial option
pricing model. It turns out that we can construct replicating portfolios in a
similar manner as before and obtain an analogous \emph{risk-neutral pricing
formula}. But there is some twist \warn{}: Due to the \emph{path-dependent}
nature, different asset price paths that result in the \emph{same} terminal asset
price may lead to \emph{different} payoffs for an Asian option!

For example, the payoffs of an Asian option can be different for the following
two paths, although they both lead to the same terminal asset price \(S_{ud}\).
This is because the \emph{intermediate} asset price can differ, which
influences the average calculation.
\begin{center}
\begin{tikzpicture}
\node[] (s0) at (0,0) {\(S_0\)};
\node[] (su) at (4,1) {\(S_u\)};
\node[] (sd) at (4,-1) {\(S_d\)};
\node[] (suu) at (8,2) {\(S_{uu}\)};
\node[] (sud) at (8,0) {\(S_{ud}\)};
\node[] (sdd) at (8,-2) {\(S_{dd}\)};
\draw[-Latex, blue] (s0.east) -- (su.west);
\draw[-Latex, blue, opacity=0.2] (s0.east) -- (sd.west);
\draw[-Latex, violet, opacity=0.2] (su.east) -- (suu.west);
\draw[-Latex, violet] (su.east) -- (sud.west);
\draw[-Latex, orange, opacity=0.2] (sd.east) -- (sud.west);
\draw[-Latex, orange, opacity=0.2] (sd.east) -- (sdd.west);
\node[] () at (0,3) {Time 0};
\node[] () at (4,3) {Time \(h\)};
\node[] () at (8,3) {Time \(2h\)};
\end{tikzpicture}
\begin{tikzpicture}
\node[] (s0) at (0,0) {\(S_0\)};
\node[] (su) at (4,1) {\(S_u\)};
\node[] (sd) at (4,-1) {\(S_d\)};
\node[] (suu) at (8,2) {\(S_{uu}\)};
\node[] (sud) at (8,0) {\(S_{ud}\)};
\node[] (sdd) at (8,-2) {\(S_{dd}\)};
\draw[-Latex, blue, opacity=0.2] (s0.east) -- (su.west);
\draw[-Latex, blue] (s0.east) -- (sd.west);
\draw[-Latex, violet, opacity=0.2] (su.east) -- (suu.west);
\draw[-Latex, violet, opacity=0.2] (su.east) -- (sud.west);
\draw[-Latex, orange] (sd.east) -- (sud.west);
\draw[-Latex, orange, opacity=0.2] (sd.east) -- (sdd.west);
\node[] () at (0,3) {Time 0};
\node[] () at (4,3) {Time \(h\)};
\node[] () at (8,3) {Time \(2h\)};
\end{tikzpicture}
\end{center}

\item Therefore, although the risk-neutral pricing formula is most similar to
before, we need to consider all the \(2^n\) asset price paths instead of just
the \(n+1\) terminal asset prices, for an \(n\)-period binomial tree. We need
to separately calculate the payoff for \emph{each} path, and multiply it by the
corresponding risk-neutral probability for travelling \emph{in this specific
path}.

For example, the risk-neutral pricing formula in a two-period binomial tree
takes the following form:
\begin{align*}
V_0&=e^{-rT}\sum_{\text{all \(2^2=4\) paths}}^{}\text{RN probability for travelling in that path}\times \text{payoff for that path}\\
&=e^{-rT}[(p^{*})^{2}V_{uu}+p^{*}(1-p^{*})V_{ud}+(1-p^{*})p^{*}V_{du}+(1-p^{*})^{2}V_{dd}].
\end{align*}
\begin{remark}
\item We need to consider the \(ud\) and \(du\) paths separately as
\(V_{ud}\ne V_{du} \) in general.
\item The RN probability for travelling in the \(ud\) path, i.e., ``up'' and
then ``down'' \emph{in this order} (\underline{not} just one ``up'' and one
``down''), is \(p^{*}(1-p^{*})\), \underline{not} \(2p^{*}(1-p^{*})\)! This is
similar for the \(du\) path.
\end{remark}
\end{enumerate}
\subsection{Barrier Options}
\label{subsect:barrier-option}
\begin{enumerate}
\item A \defn{barrier option} is another path-dependent option whose payoff
depends on whether the underlying asset price \emph{reaches} or \emph{hits} a
specified level, called the \defn{barrier}, over the life of the option. 

\item Like Asian option, a barrier option also has three dimensions:
\begin{enumerate}[label={(\arabic*)}]
\item \emph{``up'' or ``down'':} It refers to whether the asset price needs to
go \emph{up} \faIcon{arrow-up} or \emph{down} \faIcon{arrow-down} from time 0
to hit the barrier \(B\). In other words, it is an \defn{up option} (\defn{down option})
if the barrier \(B\) is higher (lower) than the current asset price
\(S_0\).
\item \emph{``in'' or ``out'':} The option can be either \emph{knocked-in} or
\emph{knocked-out}. When the barrier is hit,
\begin{itemize}
\item \defn{knocked-in}: a plain vanilla European option \emph{comes into
existence};
\item \defn{knocked-out}: the original plain vanilla European option
\emph{ceases to exist} and becomes worthless \faIcon{trash-alt}.
\end{itemize}
\begin{note}
The underlying plain vanilla option is supposed to have a fixed expiration
time, independent from the ``hitting time''.
\end{note}
\item \emph{``call'' or ``put'':} It refers to whether the underlying plain
vanilla option is a call or put option.
\end{enumerate}

\item Comparing a barrier option and an otherwise identical plain vanilla
option, we can observe that the barrier option \emph{never pays more} than the
plain vanilla option. Hence, the barrier option must be no more expensive
than the plain vanilla one. This makes barrier option as a more ``economic''
alternative to the plain vanilla one.

\item \label{it:barrier-option-payoff-fmlas} To be more precise, we can express payoff formula of a barrier option
mathematically as follows. Consider a \(T\)-year \(K\)-strike underlying
European call/put. Let \(\displaystyle M_T=\max_{t\in[0,T]}S_t\) and
\(\displaystyle m_T=\min_{t\in[0,T]}S_t\) be the maximum and minimum asset
prices in the path respectively.

Then, the time-\(T\) payoffs of the 8 kinds of barrier option are given by:
\begin{enumerate}[label={(\arabic*)}]
\item up-and-in call: \((S_T-K)_{+}\indicset{M_T\ge B}\) \begin{note}
\(M_T\ge B\) means that \(S_t\ge B\) for some time \(t\in[0,T]\)
\faIcon{arrow-right} asset price has gone high enough to hit the barrier
\faIcon{arrow-right} ``knocked-in''.
\end{note}
\item up-and-in put: \((K-S_T)_{+}\indicset{M_T\ge B}\)
\item up-and-out call: \((S_T-K)_{+}\indicset{M_T<B}\) \begin{note}
\(M_T<B\) means that \(S_t<B\) for \emph{any} time \(t\in [0,T]\)
\faIcon{arrow-right} asset price has \emph{never} gone high enough to hit the
barrier
\faIcon{arrow-right} never ``knocked-out''
\end{note}
\item up-and-out put: \((K-S_T)_{+}\indicset{M_T<B}\)
\item down-and-in call: \((S_T-K)_{+}\indicset{m_T\le B}\) \begin{note}
\(m_T\le B\) means that \(S_t\le B\) for some time \(t\in[0,T]\)
\faIcon{arrow-right} asset price has gone low enough to hit the barrier
\faIcon{arrow-right} ``knocked-in''.
\end{note}
\item down-and-in put: \((K-S_T)_{+}\indicset{m_T\le B}\)
\item down-and-out call: \((S_T-K)_{+}\indicset{m_T>B}\) \begin{note}
\(m_T>B\) means that \(S_t>B\) for \emph{any} time \(t\in [0,T]\)
\faIcon{arrow-right} asset price has \emph{never} gone low enough to hit the
barrier
\faIcon{arrow-right} never ``knocked-out''
\end{note}
\item down-and-out put: \((K-S_T)_{+}\indicset{m_T>B}\)
\end{enumerate}
\item \label{it:barrier-options-parity} A remarkable relationship between
otherwise identical knocked-in and knocked-out barrier options is known as
\emph{barrier options parity}:
\[
\text{knocked-in option price}+\text{knocked-out option price}
=\text{plain vanilla option price}.
\]
\begin{pf}
We can just check the payoff formulas for all 4 types of combinations of
otherwise identical knocked-in and knocked-out options (up/down \& call/put).
Here we only check the combination of up-and-in call and up-and-out call.
Others can be checked similarly.

In this case, according to \labelcref{it:barrier-option-payoff-fmlas}, the
payoff of the combination of up-and-in call and up-and-out call is
\[
(S_T-K)_{+}\indicset{M_T\ge B}+(S_T-K)_{+}\indicset{M_T< B}
=(S_T-K)_{+},
\]
which equals the payoff of the plain vanilla call. The parity then follows by
\emph{law of one price}.
\end{pf}
\end{enumerate}
