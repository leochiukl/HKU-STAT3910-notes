\section{Interest Rate Derivatives}
\label{sect:ir-deriv}
\begin{enumerate}
\item In \Cref{sect:ir-deriv}, we will apply the methodologies from the
binomial tree model and Black-Scholes model to price \defn{interest rate
derivatives}, which are derivatives whose payoffs depend on interest rates,
e.g., zero-coupon bonds (simplest one!), options on zero-coupon bonds, and
interest rate caps and floors.
\end{enumerate}
\subsection{The Binomial Tree Approach}
\label{subsect:ir-deriv-binomial-tree}
\begin{enumerate}
\item To apply the binomial tree model for pricing interest rate derivatives,
as one may expect, we are going to assume that (risk-free) \emph{interest
rates} (not stock prices this time) evolve in a binomial fashion. \begin{note}
The interest rates involved can be continuously compounded, effective annual,
or in some other compounding frequency. Be careful about the specification of
interest rate!
\end{note}

\item More specifically, at the end of each period, the interest rate over the
period can change to \emph{two} possible values: one corresponds to the ``up''
move and another corresponds to the ``down'' move. The new interest rate is
then applied over the next period.

Example: The following is a \emph{3-period} (\warn{}) binomial interest rate
tree where the length of each period is 1 year.
\begin{center}
\begin{tikzpicture}
\node[blue] (r0) at (0,0) {\(r_0\)};
\node[violet] (ru) at (4,1) {\(r_u\)};
\node[violet] (rd) at (4,-1) {\(r_d\)};
\node[magenta] (ruu) at (8,2) {\(r_{uu}\)};
\node[magenta] (rud) at (8,0) {\(r_{ud}\)};
\node[magenta] (rdd) at (8,-2) {\(r_{dd}\)};
\draw[-Latex] (r0.east) -- (ru.west);
\draw[-Latex] (r0.east) -- (rd.west);
\draw[-Latex] (ru.east) -- (ruu.west);
\draw[-Latex] (ru.east) -- (rud.west);
\draw[-Latex] (rd.east) -- (rud.west);
\draw[-Latex] (rd.east) -- (rdd.west);
\node[] () at (0,3) {Time \(0\)};
\node[] () at (4,3) {Time \(1\)};
\node[] () at (8,3) {Time \(2\)};
\draw[<->, blue] (0,4) -- (4,4);
\node[blue] () at (2,4.5) {period 1};
\draw[<->, violet] (4,4) -- (8,4);
\node[violet] () at (6,4.5) {period 2};
\draw[<->, magenta] (8,4) -- (12,4);
\node[magenta] () at (10,4.5) {period 3};
\end{tikzpicture}
\end{center}
\begin{itemize}
\item The interest rate over \blc{period 1} is \blc{\(r_0\)} (non-random).
\item The interest rate over \vc{period 2} is \vc{\(r_u\) or \(r_d\)} (random).
\item The interest rate over \mgc{period 3} is \mgc{\(r_{uu}\), \(r_{ud}\), or \(r_{dd}\)} (random).
\end{itemize}
Apart from this special feature that the value at each node applies for one
period (but not just for a single time point), another special feature for
binomial interest rate tree is that the \emph{risk-neutral probability} of an
up move \(p^{*}\) is \emph{not} directly computed based on the tree parameters,
but is rather an \emph{additional} tree parameter to be specified. \begin{warning}
We do \underline{not} have \(p^{*}=\frac{e^{(r-\delta)h}-d}{u-d}\) in a
binomial tree \emph{interest rate} tree!
\end{warning}
\begin{note}
Roughly speaking, this situation arises since unlike a stock, interest rate is
not tradable directly, so we are unable to just obtain the risk-neutral
probability simply by a replicating portfolio argument (like what we did for
binomial stock price tree).
\end{note}

\item Given a binomial interest rate tree together with the additional
parameter \(p^*\), we can price interest rate derivatives using the
\emph{risk-neutral pricing} method, similar to what we did before for stock
options, but with some subtleties \warn{}.

Here, the interest rates are used for \emph{two} purposes:
\begin{enumerate}[label={(\arabic*)}]
\item defining the payoff of the interest rate derivative
\item discounting the payoff from the time of payment to time 0
\end{enumerate}
Notably, interest rate derivatives are always \emph{path-dependent} in the
sense that the discounted payoffs for different paths would be different in
general, since at least the discounting involved would be different (even if
the payoffs are the same).

\item \label{it:ir-deriv-rn-pricing}
By risk-neutral pricing\footnote{We shall not discuss the justification
of risk-neutral pricing in this context as it would involve technical concepts
to be discussed in STAT3911.}, the time-0 price of an interest rate derivative
is
\[
V_0=\boxed{\sum_{\text{all paths}}^{}\text{discount factor}\times \text{RN probability}\times \text{payoff}}.
\]
Note that ``discount factor'' here generally differs for different interest
rate paths, and thus needs to be placed inside the summation (unlike the case
for binomial stock price trees).

Here, we will discuss three examples of interest rate derivatives for
illustrate the usage of such risk-neutral pricing formula:
\begin{enumerate}[label={(\arabic*)}]
\item zero-coupon bonds
\item options on zero-coupon bonds
\item interest rate caps and floors
\end{enumerate}

\item \label{it:zcb-rn-pricing-fmla}
Let us start with the simplest interest rate derivative: (risk-free)
zero-coupon bonds (ZCBs). We have already learnt what a zero-coupon bond is in
STAT2902: It (only) pays a fixed amount, which is known as the face value, at
the maturity date of the bond. In the case where the face value equals 1, by
risk-neutral pricing, the time-0 bond price is
\[
B_0=\boxed{\sum_{\text{all paths}}^{}\text{discount factor}\times \text{RN probability}}.
\]
since \(\text{payoff}\equiv\text{face value}=1\).

\begin{note}
In general, when the face value of the zero-coupon bond is \(F\), its time-0
price is just given by \(F\times B_0\) (as we have \(\text{payoff}\equiv F\)
instead).
\end{note}

In STAT2902, we have learnt the pricing of zero-coupon bonds in a simpler
setting where the interest rate is a known constant. Here we are in a slightly
more complex setting where the interest rates evolve according to a binomial
tree, and hence we need to do the discounting on a \emph{path-by-path} basis.

\item \label{it:zcb-option-rn-pricing-fmla}
After warming up with pricing zero-coupon bonds, we now consider some
more complicated interest rate derivatives. First we discuss the pricing of
options on zero-coupon bonds. Consider a \(K\)-strike \(T\) year European call
option on a \(s\)-year zero-coupon bond paying \(1\) (i.e., the zero-coupon
bond matures at time \(T+s\), not time \(s\) \warn{}).

\begin{center}
\begin{tikzpicture}
\draw[-Latex] (0,0) -- (8,0)
node[right]{Time};
\draw[fill] (0,0) circle [radius=0.5mm];
\node[] () at (0,-0.5) {\(0\)};
\node[] () at (0,-1) {buy call};
\draw[fill] (4,0) circle [radius=0.5mm];
\node[] () at (4,-0.5) {\(T\)};
\node[] () at (4,-1) {call expires};
\draw[fill] (7,0) circle [radius=0.5mm];
\node[] () at (7,-0.5) {\(T+s\)};
\node[] () at (7,-1) {ZCB expires};
\draw[very thick, decorate,decoration={calligraphic brace, amplitude=5pt, raise=5pt}] (4,0) -- (7,0)
node[midway, above=5mm]{\(s\) years};
\end{tikzpicture}
\end{center}
The time-\(T\) payoff of such call option is \((P(T,T+s)-K)_{+}\) where
\(P(T,T+s)\) is the time-\(T\) price of the underlying ZCB that matures at time
\(T+s\). It is a random variable, viewing from time 0. So, plugging this into
the risk-neutral pricing formula, the time-0 price of the call is
\[
C_0=\boxed{\sum_{\text{all paths}}^{}\text{discount factor}\times \text{RN probability}\times (P(T,T+s)-K)_{+}}.
\]
For the otherwise identical \emph{put} option, its time-0 price is
\[
P_0=\boxed{\sum_{\text{all paths}}^{}\text{discount factor}\times \text{RN probability}\times (K-P(T,T+s))_{+}}.
\]
\item Lastly, we consider interest rate caps and floors. They are formed by
some building blocks known as interest rate \emph{caplets} and
\emph{floorlets}, which are similar to call and put options on the interest
rate, respectively.

\item An \defn{interest rate caplet} is characterized by 3 elements:
\begin{enumerate}[label={(\arabic*)}]
\item \blc{\emph{a fixed single time period:}} we are focusing on interest rates over this period
\item \vc{\emph{cap rate:}} a fixed interest over the period, serving as a ``cap'' on the actual interest payment
\item \brc{\emph{notional amount:}} the amount you borrowed ``notionally'', which is
used for the ``notional'' calculations of interest payments
\end{enumerate}
For illustration purpose, let us consider a concrete example about interest
rate caplet. Suppose that you have borrowed 100 at a floating rate with annual
interest rate payments (made at the end of each year), i.e., the interest rate
charged ``floats'' according to the market interest rate. Assume that the
effective interest rate over this year is \(8\%\), which is supposed to be
known and fixed at the beginning of this year. Then, at the end of this year,
you will need to pay an interest of \(100\times 8\%=8\).

Note that here you are exposed to the upside interest rate risk. If the
realized interest rate over this year were abnormally high (say \(1000\%\)),
you would need to pay a huge amount of interest (\(100\times 1000\%=1000\)
\warn{}).  To protect against this risk, an interest rate caplet that ``caps''
the actual interest payment can be purchased.

Suppose that you have purchased a \vc{5\%} (effective annual) interest rate
caplet on a \brc{100} loan over \blc{this year}. Knowing that the interest rate
over this year is 8\% at the beginning, which exceeds the cap rate \vc{5\%},
there would be a caplet payment made to you \emph{at the beginning} of this
year, of amount
\[
\frac{\brc{100}\times (8\%-\vc{5\%})}{1+8\%}.
\]
This payment amount can insure against the interest rate risk, since if you
invest this amount at the 8\% (risk-free) interest rate over this year, you can
receive \(\brc{100}\times (8\%-\vc{5\%})\) at the end of the year. Hence, your
net payment at the end of the year will just be
\[
\underbrace{\brc{100}\times 8\%}_{\text{original interest payment}}-\underbrace{\brc{100}\times
(8\%-\vc{5\%})}_{\text{return from investment}}
=\brc{100}\times \vc{5\%}=5,
\]
so it is like the interest payment is capped at the amount required under 5\%
interest rate.

On the other hand, if the effective interest rate over this year is
\emph{lower} than 5\%, then there would not be any caplet payment.

So, in general, the caplet payment, made at the beginning of \blc{the period},
is given by
\[
\brc{\text{notional amount}}\times \frac{(\text{interest rate}-\vc{\text{cap rate}})_{+}}{1+\text{interest rate}}
\]
(assuming the interest rates are all effective annual).

\item After understanding interest rate caplet, an \defn{interest rate
floorlet} works analogously, and is again characterized by 3 elements:
\begin{enumerate}[label={(\arabic*)}]
\item \blc{\emph{a fixed single time period:}} we are focusing on interest rates over this period
\item \vc{\emph{floor rate:}} a fixed interest over the period, serving as a ``floor'' on the actual interest payment
\item \brc{\emph{notional amount:}} the amount you lent ``notionally'', which is
used for the ``notional'' calculations of interest payments
\end{enumerate}
An interest rate floorlet is used for another purpose: hedging against the
interest rate risk for the \emph{lender}. Continuing from the example above,
now suppose that you are the lender of the loan rather than the borrower.

This time, you are exposed to the downside interest rate risk. If the realized
interest rate over this period were abnormally low (say 0.001\%), you would only
receive a tiny amount of interest (\(100\times 0.001\%=0.001\) \warn{}). To
protect against this risk, you can purchase an interest rate floor.

Suppose that you have purchased a \vc{5\%} (effective annual) interest rate
caplet on a \brc{100} loan over \blc{this year}. Assuming that the interest
rate over this year is known to be 2\% at the beginning, which is below the cap
rate \vc{5\%}, there would be a floor payment made to you \emph{at the
beginning} of this year, of amount
\[
\frac{\brc{100}\times (\vc{5\%}-2\%)}{1+2\%}.
\]
Similarly, after investing this amount at 2\% interest rate over this year, your
net amount to be received at the end of the year will be
\[
\brc{100}\times 2\%+\brc{100}\times (\vc{5\%}-2\%)
=\brc{100}\times \vc{5\%}=5.
\]

In general, the floor payment, made at the beginning of \blc{the period}, is
given by
\[
\brc{\text{notional amount}}\times \frac{(\vc{\text{cap rate}}-\text{interest rate})_{+}}{1+\text{interest rate}}
\]
(assuming the interest rates are all effective annual).

\item \label{it:ir-cap-floor-rn-pricing-fmla} Each interest rate
caplet/floorlet is only over a single time period. To hedge against interest
rate risk over multiple time periods, we can purchase multiple interest rate
caplets/floorlets, and a collection of interest rate caplets (floorlets) over
different time periods is known as a \defn{interest rate cap} (\defn{interest
rate floor}).

Since an interest rate cap (floor) comprises of multiple interest rate
caplets (floorlets), there would be multiple potential caplet (floorlet)
payments made at different time points also. Hence, to price an interest rate
cap (floor) in a binomial interest rate tree, it is not enough to just look at
the payoffs at terminal nodes --- we also need to consider the earlier nodes
(somewhat like American options...), as payments may arise at those earlier
nodes as well!

So, by risk-neutral pricing, the time-0 price of an interest rate cap is the
sum of the prices of all the caplets in the composition:
\[
V_0=\boxed{\sum_{\text{all paths to nonzero caplet payment}}^{}\text{discount factor}\times \text{RN probability}\times \text{caplet payment}}.
\]
\begin{warning}
It is possible that a single caplet payment is triggered by \emph{multiple}
paths! For example, in case there is a caplet payment at the \(ud\) node, then
there would be 2 paths reaching the \(ud\) node, where the discounting is done
differently.
\end{warning}

We replace ``caplet'' by ``floorlet'' in the formula for an interest rate
floor.

\item Example: Consider the following binomial interest rate tree with
\(p^*=1/2\) (where all interest rates are effective annual rates):
\begin{center}
\begin{tikzpicture}
\node[] (r0) at (0,0) {\(r_0=6\%\)};
\node[] (ru) at (4,1) {\(r_u=7.704\%\)};
\node[] (rd) at (4,-1) {\(r_d=4.673\%\)};
\node[] (ruu) at (8,2) {\(r_{uu}=9.892\%\)};
\node[] (rud) at (8,0) {\(r_{ud}=6\%\)};
\node[] (rdd) at (8,-2) {\(r_{dd}=3.639\%\)};
\draw[-Latex] (r0.east) -- (ru.west);
\draw[-Latex] (r0.east) -- (rd.west);
\draw[-Latex] (ru.east) -- (ruu.west);
\draw[-Latex] (ru.east) -- (rud.west);
\draw[-Latex] (rd.east) -- (rud.west);
\draw[-Latex] (rd.east) -- (rdd.west);
\node[] () at (0,3) {Time \(0\)};
\node[] () at (4,3) {Time \(1\)};
\node[] () at (8,3) {Time \(2\)};
\end{tikzpicture}
\end{center}
Suppose that we want to price a \vc{7.5\%} interest rate cap on a \brc{100}
\emph{3-year} loan.

The first step is to understand the composition of such interest rate cap.
Here, it consists of 3 interest rate caplets:
\begin{enumerate}
\item 7.5\% interest rate caplet on a 100 loan over first year (time 0 to time
1) \faIcon{arrow-right} potential caplet payment at time 0 (initial node)
\item 7.5\% interest rate caplet on a 100 loan over second year (time 1 to time 2)
\faIcon{arrow-right} potential caplet payment at time 1 (\(u\) or \(d\) node)
\item 7.5\% interest rate caplet on a 100 loan over third year (time 2 to time 3)
\faIcon{arrow-right} potential caplet payment at time 2 (\(uu\), \(ud\) or \(du\) node)
\end{enumerate}

Next, since the caplet payment is nonzero only when the realized effective
interest rate exceeds 7.5\%, there would only be nonzero caplet payments at the
\(u\) node and \(uu\) node.

Then, we calculate the caplet payments at \(u\) and \(uu\) nodes:
\begin{itemize}
\item \emph{\(u\) node:}
\[
\frac{100\times (7.704\%-7.5\%)}{1+7.704\%}=0.189408.
\]
\begin{note}
The realized interest rate \(r_u=7.704\%\) is applied for the year starting at
time 1, i.e., the second year, and the caplet payment here is made at the
beginning of second year, i.e., time 1.
\end{note}
\item \emph{\(uu\) node:}
\[
\frac{100\times (9.892\%-7.5\%)}{1+9.892\%}=2.176683.
\]
\end{itemize}
\begin{center}
\begin{tikzpicture}
\node[] (r0) at (0,0) {\(r_0=6\%\)};
\node[] (ru) at (4,1) {\(r_u=7.704\%\)};
\node[] (rd) at (4,-1) {\(r_d=4.673\%\)};
\node[] (ruu) at (8,2) {\(r_{uu}=9.892\%\)};
\node[] (rud) at (8,0) {\(r_{ud}=6\%\)};
\node[] (rdd) at (8,-2) {\(r_{dd}=3.639\%\)};
\node[draw] () at (4,1.5) {\faIcon{dollar-sign}: +0.189408};
\node[draw] () at (8,2.5) {\faIcon{dollar-sign}: +2.176683};
\draw[-Latex] (r0.east) -- (ru.west);
\draw[-Latex] (r0.east) -- (rd.west);
\draw[-Latex] (ru.east) -- (ruu.west);
\draw[-Latex] (ru.east) -- (rud.west);
\draw[-Latex] (rd.east) -- (rud.west);
\draw[-Latex] (rd.east) -- (rdd.west);
\node[] () at (0,3) {Time \(0\)};
\node[] () at (4,3) {Time \(1\)};
\node[] () at (8,3) {Time \(2\)};
\end{tikzpicture}
\end{center}
Finally, by risk-neutral pricing, the time-0 price of the interest rate cap is
\[
V_0=(1+r_0)^{-1}\times p^*\times 0.189408+(1+r_0)^{-1}(1+r_u)^{-1}\times (p^*)^{2}\times 2.176683
=0.5660.
\]
\end{enumerate}
\subsection{The Black-Scholes Approach}
\begin{enumerate}
\item After discussing the pricing of interest rate derivatives using the
binomial tree model in \Cref{subsect:ir-deriv-binomial-tree}, we will discuss
how to perform the pricing under the Black-Scholes model. Here, we need to
assume a certain price process follows geometric Brownian motion. But what
price process should be assumed to follow a geometric Brownian motion?

\item It turns out that we are assuming that the \emph{bond forward price}
(i.e., the delivery/forward price of a forward contract on a zero-coupon bond)
follows a geometric Brownian motion. Before explaining the rationale behind
this assumption, we will have some discussions about the bond forward price.

Throughout, we shall use the (seemingly complex, but useful) notation
\(P_t(T,T+s)\) to denote the time-\(t\) forward price (i.e., the delivery price
negotiated at time \(t\)) of a forward on a ZCB paying \(1\) at time \(T+s\),
to be delivered at time \(T\).
\begin{center}
\begin{tikzpicture}
\draw[-Latex] (0,0) -- (8,0)
node[right]{Time};
\draw[fill] (1,0) circle [radius=0.5mm];
\node[] () at (1,-0.5) {\(t\)};
\node[] () at (1,0.5) {forward price: \(P_t(T,T+s)\)};
\node[] () at (1,-1) {current time};
\draw[fill] (4,0) circle [radius=0.5mm];
\node[] () at (4,-0.5) {\(T\)};
\node[text width=2.5cm] () at (4,-1) {delivery time of forward};
\draw[fill] (7,0) circle [radius=0.5mm];
\node[] () at (7,-0.5) {\(T+s\)};
\node[] () at (7,-1) {ZCB expires};
\draw[very thick, decorate,decoration={calligraphic brace, amplitude=5pt, raise=5pt}] (4,0) -- (7,0)
node[midway, above=5mm]{\(s\) years};
\end{tikzpicture}
\end{center}
More specifically, after entering into such forward at time \(t\), we are
obligated to pay the forward price \(P_t(T,T+s)\) to get the ZCB paying \(1\)
at time \(T+s\), at time \(T\).

\item In the special case where \(t=T\), the time-\(t\) ``forward price''
\(P_t(T,T+s)=P_t(t,T+s)\) is nothing but just the time-\(t\) spot price of the
ZCB paying 1 at time \(T+s\), since under the absence of arbitrage
opportunities, the only possible price negotiated is the time-\(t\) spot
price.

\item \label{it:bond-fwd-price-ratio-fmla} Under the no-arbitrage principle, we
have the following ``ratio formula'' for computing bond forward price:
\[
\boxed{P_t(T,T+s)=\frac{P_t(t,T+s)}{P_t(t,T)}}
\]
for any \(t\in[0,T]\). Rearranging the equation gives
\(P_t(t,T+s)=P_t(t,T)\times P_t(T,T+s)\), which looks like the well-known
formula \(\px[t+u]{x}=\px[t]{x}\times \px[u]{x+t}\) we learn in STAT3901!

\begin{pf}
The main idea is to construct two strategies to get the ZCB paying \(1\) at time
\(T+s\) (corresponding to \(P_t(t,T+s)=P_t(t,T)\times P_t(T,T+s)\)) and use the
law of one price to equate their time-\(t\) values. The two strategies are as
follows:
\begin{enumerate}[label={(\arabic*)}]
\item at time \(t\) (current time): buying the ZCB paying \(1\) at time \(T+s\), at the time-\(t\) spot price
\item at time \(t\) (current time): buying a ZCB paying \(P_t(T,T+s)\) at time
\(T\), and entering into the forward on ZCB (then at time \(T\), the payment
from the ZCB will be just enough for paying the delivery price)
\end{enumerate}
The time-\(t\) value of (1) is \(P_t(t,T+s)\), and the time-\(t\) value of (2)
is \(P_t(t,T)\times P_t(T,T+s)\).
\end{pf}

\item Now, we consider a \(T\)-year \(K\)-strike European call option on an
\(s\)-year ZCB paying \(1\) (i.e., ZCB expires at time \(T+s\)). The time-\(T\)
payoff this call is
\[
(P_T(T,T+s)-K)_{+}
=\qty(P_T(T,T+s)-K\times \underbrace{P_T(T,T)}_{1})_{+}
\triangleq (S_T^{(1)}-S_T^{(2)})+
\]
After writing the payoff in such a way, it becomes clear that the call on ZCB can
indeed be viewed as an \emph{exchange option}. More specifically, we let:
\begin{itemize}
\item asset 1 (underlying asset): a ZCB of \(1\) maturing at time \(T+s\)
(time-\(T\) price: \(S_T^{(1)}=P_T(T,T+s)\), which is random)
\item asset 2 (strike asset): a ZCB of \(K\) maturing at time \(T\) (time-\(T\)
price: \(S_T^{(2)}=K\times P_T(T,T)\), which is nonrandom)
\end{itemize}
Then, the call option can be seen as an exchange option which gives us the
right to exchange asset 2 for asset 1.

\begin{remark}
\item The time-\(t\) (spot) prices of assets 1 and 2 are
\(S_t^{(1)}=P_t(t,T+s)\) and \(S_t^{(2)}=K\times P_t(t,T)\).
\item Both assets 1 and 2 here are nondividend-paying, since ZCB does not pay
any dividends.
\end{remark}

\item This point of view motivates us to use the pricing assumption as
suggested in \labelcref{it:ex-opt-assum} for pricing options on ZCB, namely
assuming that the ratio of the two asset prices \(\displaystyle
\qty{\frac{S_t^{(1)}}{S_t^{(2)}}}_{t\in[0,T]}=\qty{\frac{P_t(t,T+s)}{K\times P_t(t,T)}}_{t\in[0,T]}\) is a
geometric Brownian motion (GBM) with volatility parameter \(\sigma\).

The expression \(\displaystyle \frac{P_t(t,T+s)}{K\times P_t(t,T)}\) appears to be quite
familiar. Indeed, it is just \(P_t(T,T+s)/K\) by
\labelcref{it:bond-fwd-price-ratio-fmla}.

Since \(\{P_t(T,T+s)/K\}\) follows a GBM iff \(\{P_t(T,T+s)\}\) follows a GBM,
and both GBMs would have the same volatility parameter \(\sigma\)\footnote{This
can be seen by noting that \(\vari{\ln(P_t(T,T+s)/K)}=\vari{\ln P_t(T,T+s)-\ln
K}=\vari{\ln P_t(T,T+s)}\).}, the assumption above is just equivalent to the
assumption that the \emph{bond forward price process}
\(\{P_t(T,T+s)\}_{t\in[0,T]}\) follows a geometric Brownian motion with
volatility \(\sigma\), where \(\sigma\) is the volatility of the bond forward:
\[
\vari{\ln P_t(T,T+s)}=\sigma^2 t,
\]
for any \(t\in[0,T]\). Usually the pricing assumption is phrased in this way
(i.e., in terms of bond forward price).

\item \label{it:zcb-call-put-bs-pricing-fmla} With this pricing assumption, we can
apply the formulas derived in \Cref{subsect:exchange-opt} to compute the time-0
prices of European options on ZCBs:
\begin{itemize}
\item \emph{\(T\)-year \(K\)-strike call on ZCB:}
\begin{align*}
C_0&=\bs{S^{(1)}_{0},0}{S^{(2)}_{0},0}{\sigma,T} \\
&=\boxed{\bs{P_0(0,T+s),0}{K\times P_0(0,T),0}{\sigma,T}} \\
&=P_0(0,T+s)\Phi(d_1)-K\times P_0(0,T)\Phi(d_2)
\end{align*}
where
\[
d_1=\frac{\ln[P_0(0,T+s)/(K\times P_0(0,T))]+\frac{1}{2}\sigma^2 T}{\sigma\sqrt{T}}
\qqtext{and}
d_2=d_1-\sigma\sqrt{T}.
\]
\item \emph{\(T\)-year \(K\)-strike put on ZCB:}
\begin{align*}
P_0&=\bs{S^{(2)}_{0},0}{S^{(1)}_{0},0}{\sigma,T} \\
&=\boxed{\bs{K\times P_0(0,T),0}{P_0(0,T+s),0}{\sigma,T}} \\
&=K\times P_0(0,T)\Phi(d_1)-P_0(0,T+s)\Phi(d_2)
\end{align*}
where
\[
d_1=\frac{\ln[(K\times P_0(0,T))/P_0(0,T+s)]+\frac{1}{2}\sigma^2 T}{\sigma\sqrt{T}}
\qqtext{and}
d_2=d_1-\sigma\sqrt{T}.
\]
\end{itemize}
\end{enumerate}
