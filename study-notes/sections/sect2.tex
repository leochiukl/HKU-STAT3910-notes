\section{Black-Scholes Model}
\label{sect:bs-model}
\begin{enumerate}
\item In STAT3905, we have studied Black-Scholes model. It will be studied
again in STAT3910, but in more details. We will consider a more general form of
Black-Scholes model involving the notion of \emph{Brownian motion}, something
we have learnt in STAT3903. But it turns out that this is still not the
``real'' form.  We will only cover it in full mathematical details in STAT3911.
\end{enumerate}
\subsection{Model Formulation}
\begin{enumerate}
\item\label{it:bs-model-formulation} In the \defn{Black-Scholes model} (or
\defn{Black-Scholes framework}), we are assumed to be in a perfect market
having the following two assets:
\begin{itemize}
\item a risky stock \faIcon{apple-alt} which pays dividend continuously
at a dividend yield \(\delta\), where its time-\(t\) price is
\[
S_t=S_0\exp\qty[\qty(\alpha-\delta-\frac{\sigma^2}{2})t+\sigma W_t],
\]
for some nonnegative parameters \(\alpha\) and \(\sigma\), and \(\{W_t\}\) is a
standard Brownian motion or Wiener process;
\begin{note}
The stock prices form another stochastic process \(\{S_t\}\), which is called
\defn{geometric Brownian motion}.
\end{note}
\item a risk-free zero-coupon bond \faIcon{file-invoice-dollar} with an
annual continuously compounded risk-free rate \(r\).
\end{itemize}
Recall from STAT3903 that a \emph{Wiener process} \(\{W_t\}\) (\(t\ge 0\)) is
defined by the following properties:
\begin{enumerate}
\item (starting at \(0\)) \(W_0=0\).
\item (continuity) \(W_t\) is continuous function of time \(t\).
\item (independent increments) Fix any \(n\) time points with
\(0<t_1<t_2<\dotsb<t_n\). Then the increments
\[
W_{t_1}=W_{t_1}-\underbrace{W_0}_{0},\quad W_{t_2}-W_{t_1},\qqtext{...},\quad
W_{t_n}-W_{t_{n-1}}
\]
over non-overlapping time intervals are independent random variables.
\item (stationary increments) For any \(s\ge 0\) and any \(t\ge 0\),
\(W_{s+t}-W_{s}\sim \ndist{0}{t}\), so the distribution depends only on the time
length \(t\) and is free of \(s\).
\end{enumerate}

\item In this more general characterization of Black-Scholes model, the
parameters \(\alpha\) and \(\sigma\) still carry the same meanings as before.

Interpretations of \(\alpha\) and \(\sigma\):
\begin{itemize}
\item \(\alpha\) is the \emph{continuously compounded expected rate of return
on the stock \faIcon{apple-alt}}:
\begin{align*}
\expv{S_t}&=\exp\qty[\ln S_0+\qty(\alpha-\delta-\frac{\sigma^2}{2})t]\expv{e^{\sigma W_t}}\\
&=\exp\qty[\ln S_0+\qty(\alpha-\delta-\frac{\sigma^2}{2})t]\exp\qty(\frac{\sigma^2 t}{2})\\
&=S_0e^{(\alpha-\delta)t}.
\end{align*}
\begin{note}
By setting \(s=0\) in the stationary increment property, we know
\(W_t=W_t-W_0\sim \ndist{0}{t}\). Now, recalling the formula of \emph{moment
generating function} for normal distribution, we have
\[
\expv{e^{\sigma W_t}}=\exp\qty(\frac{\sigma^2 t}{2}).
\]
\end{note}

This means
\[
\underbrace{S_0}_{\mathclap{\text{beginning value}}}e^{\alpha t}
=\underbrace{e^{\delta t}\expv{S_t}}_{\text{expected ending value}}.
\]
\item \(\sigma\) is the \emph{volatility of the stock \faIcon{apple-alt}}:
\[
\vari{\ln\frac{S_t}{S_0}}
=\vari{\text{non-random constant}+\sigma W_t}
=\vari{\sigma W_t}
=\sigma^2t.
\]
\end{itemize}
\item Now, we shall show that the model studied in STAT3905 is indeed a special
case of the one we study here. First fix any time interval \([s,s+t]\).  Then
note that
\[
\ln\frac{S_{s+t}}{S_s}=\ln(\frac{S_{s+t}}{S_0}\cdot\frac{S_0}{S_s})
=\qty(\alpha-\delta-\frac{\sigma^2}{2})t+\sigma(W_{s+t}-W_s).
\]
By the stationary increments property, we have \(W_{s+t}-W_{s}\sim \ndist{0}{t}\), so
we indeed have
\[
\ln\frac{S_{s+t}}{S_s}\sim \ndist{\qty(\alpha-\delta-\frac{\sigma^2}{2})t}{\sigma^{2}t}.
\]
We can also express this in terms of log-normal distribution:
\[
\frac{S_{s+t}}{S_s}\sim \lndist{\qty(\alpha-\delta-\frac{\sigma^2}{2})t}{\sigma^{2}t}.
\]

To reduce this to the case for STAT3905, we can set \(s=0\), which yields
\[
\frac{S_{t}}{S_0}\sim \lndist{\qty(\alpha-\delta-\frac{\sigma^2}{2})t}{\sigma^{2}t},
\]
or
\[
S_t\sim \lndist{\ln S_0+\qty(\alpha-\delta-\frac{\sigma^2}{2})t}{\sigma^{2}t},
\]
which is the same as the one for STAT3905, from the perspective of marginal
distribution.

\item \label{it:bs-indpt-price-ratios}
But the advantage of using the more general Black-Scholes model here is
that we can also describe the relationship between different stock prices ---
we can model the ``joint'' behaviour of stock prices. Consider \(n\) time
points with \(0<t_1<t_2<\dotsb<t_n\). Then the stock price ratios
\[
\frac{S_{t_1}}{S_{0}},\quad\frac{S_{t_2}}{S_{t_1}},\qqtext{...},\quad\frac{S_{t_n}}{S_{t_{n-1}}}
\]
are functions of the independent increments \(W_{t_1}-W_0\), \(W_{t_2}-W_{t_1}\),
..., \(W_{t_n}-W_{t_{n-1}}\). Thus, these stock price ratios over
non-overlapping time intervals are independent also.
\end{enumerate}
\subsection{Probabilistic Quantities Under Black-Scholes Model}
\begin{enumerate}
\item Since the marginal distribution of the more general Black-Scholes model
here is just the lognormal distribution we have studied in STAT3905, the
formulas for probabilistic quantities introduced in STAT3905 are also
applicable here:
\begin{enumerate}
\item\label{it:bs-ex-prob} \emph{(true) exercise probability:} (call)
\(\prob{S_T>K}=\boxed{\Phi(\widehat{d_2})}\); (put)
\(\prob{S_T<K}=\boxed{\Phi(-\widehat{d_2})}\), where
\[
\widehat{d_2}=\frac{\ln(S_0/K)+(\alpha-\delta-\sigma^2/2)T}{\sigma\sqrt{T}}.
\]
\item \label{it:bs-mean-var-stock-price} \emph{mean and variance of \(S_t\)
(lognormal mean and variance formulas):}
\(\expv{S_t}=\boxed{S_0e^{(\alpha-\delta)t}}\) and
\(\vari{S_t}=\boxed{(\expv{S_t})^{2}(e^{\sigma^{2} t}-1)}\).
\end{enumerate}

\item Here we introduce several more probabilistic quantities. The first one is
\emph{quantiles}. For the case of stock price \(S_t\) here, the \(p\)th
quantile, or \(100p\)th percentile, of \(S_t\) is just the value of \(c\) that
makes \(\prob{S_t\le c}=\alpha\). We shall denote it by
\(\pi_{p}(S_t)\).

\begin{note}
In the financial or risk management context, the quantiles are also known as
\emph{value at risk} (VaR).
\end{note}

\item \label{it:bs-stock-price-quantile-fmla} The formula for \(\pi_{p}(S_t)\)
is given by
\[
\pi_{p}(S_t)=\boxed{S_0\exp\qty[\qty(\alpha-\delta-\frac{\sigma^2}{2})t + \sigma\sqrt{t}\cdot \pi_{p}(Z)]}
\]
where \(\pi_{p}(Z)\) is the \(100p\)th percentile of a \emph{standard normal}
random variable \(Z\).

\begin{pf}
We shall use without proof the following general result: If \(g\) is an
\emph{increasing} continuous function applied to a random variable \(X\), then
\(\pi_{p}(g(Y))=g(\pi_{p}(Y))\) --- We can ``interchange'' \(\pi_{p}\) and
\(g\).

First, note that for a Weiner process \(\{W_t\}_{t\ge 0}\), for any \(t\ge 0\),
we can write \(W_t\eqd\sqrt{t}Z\) where \(Z\) is a standard normal random
variable, since \(W_t\sim N(0,t)\). Thus, we have
\[
S_t\eqd S_0\exp\qty[\qty(\alpha-\delta-\frac{\sigma^2}{2})t+\sigma\sqrt{t}Z].
\]
With a fixed \(t\ge 0\), we define the function \(g\) by
\[
g(\vc{x})=S_0\exp\qty[\qty(\alpha-\delta-\frac{\sigma^2}{2})t+\sigma\sqrt{t}\vc{x}],
\]
which is continuous and increasing. Then, applying the general result, we get
\[
\pi_p(S_t)=\pi_p(g(Z))=g(\pi_p(Z))=S_0\exp\qty[\qty(\alpha-\delta-\frac{\sigma^2}{2})t + \sigma\sqrt{t}\cdot \pi_{p}(Z)].
\]
\end{pf}
\item Next we will consider quantities involving multiple stock prices at
different time points. For these quantities, knowledge of \emph{marginal}
distributions of stock prices is not enough. We need to know their \emph{joint}
distribution. This is a place where the more general characterization of
Black-Scholes model here is useful.

\item Examples:
\begin{enumerate}
\item \emph{Probability involving multiple stock prices:} We write
\begin{align*}
\prob{S_1<S_2<S_3}
&=\prob{S_1<S_2\text{ and }S_2<S_3} \\
&=\prob{\frac{S_2}{S_1}>1\text{ and }\frac{S_3}{S_2}>1} \\
&=\prob{\frac{S_2}{S_1}>1}\prob{\frac{S_3}{S_2}>1}&\text{(independent price ratios)}.
\end{align*}
Then we can carry out the remaining computations based on the property
\[
\frac{S_{s+t}}{S_s}\sim \lndist{\qty(\alpha-\delta-\frac{\sigma^2}{2})t}{\sigma^{2}t}.
\]
\item \emph{Geometric average of stock prices:} We write
\begin{align*}
G_3&=(S_1S_2S_3)^{1/3}\\
&=S_0\exp[\qty(\alpha-\delta-\frac{1}{2}\sigma^2)\qty(\frac{1+2+3}{3})+\sigma\qty(\frac{W_1+W_2+W_3}{3})] \\
&=S_0\exp[a+b(W_1+W_2+W_3)]
\end{align*}
where \(a\) and \(b\) are some constants. Then, we write
\[
W_1+W_2+W_3=3W_1+2(W_2-W_1)+(W_3-W_2).
\]
\begin{note}
The idea for obtaining this expression is as follows.

To apply the independent increment property of Weiner process, we first write
\[
W_1+W_2+W_3=aW_1+b(W_2-W_1)+c(W_3-W_2)
\]
and we would like to find the values of \(a\), \(b\), and \(c\). Comparing the
coefficient of \(W_3\), we conclude that \(c=1\). Next, comparing the
coefficient of \(W_2\) gives \(b=2\). Finally, comparing the coefficient of
\(W_1\) gives \(a=3\).
\end{note}

Since \(W_1,W_2-W_1,W_3-W_2\iid\ndist{0}{1}\), we have
\[
W_1+W_2+W_3\sim\ndist{0}{(1^{2}+2^{2}+3^{2})(1)}\equiv\ndist{0}{14}.
\]
From this, we can calculate probabilistic quantities regarding \(G_3\).
\end{enumerate}
\end{enumerate}
\subsection{Risk-Neutral Pricing}
\begin{enumerate}
\item After discussing the formulation of the Black-Scholes model, we would
like to \emph{price} options under the Black-Scholes model, just like the case
for binomial option pricing model. Similarly, we will utilize
\emph{risk-neutral pricing}. Entering into the risk-neutral world provides us a
convenient way to price options.

\item The stock price formula in \labelcref{it:bs-model-formulation} describes
the behaviour of stock prices in the \emph{real} world.
\[
S_t=S_0\exp\qty[\qty(\alpha-\delta-\frac{\sigma^2}{2})t+\sigma W_t],\quad t\ge 0.
\]
\begin{itemize}
\item The parameter \(\alpha\) describes the continuously compounded expected
rate of return on the stock in the \emph{real} world. To compensate for the
risk inherent to the stock, the return rate \(\alpha\) is usually higher than
the risk-free rate \(r\).
\item The stochastic process \(\{W_t\}\) is a Weiner process in the \emph{real}
world.
\end{itemize}
\item Next, we want to enter into the \emph{risk-neutral} world. Before that,
we do some preparatory work. First define another stochastic process
\(\{\widetilde{W}_t\}\) by \[
 \widetilde{W}_t=W_t+\frac{\alpha-r}{\sigma}t
\]
for any \(t\ge 0\). Then, we can write the stock price formula as:
\begin{align*}
S_t&=S_0\exp\qty[\qty(\alpha-\delta-\frac{\sigma^2}{2})t+\sigma W_t] \\
&=S_0\exp\qty[\qty(\alpha-\delta-\frac{\sigma^2}{2})t+\sigma\qty(\widetilde{W}_t-\frac{\alpha-r}{\sigma}t)] \\
&=S_0\exp\qty[\qty(\alpha-\delta-\frac{\sigma^2}{2})t-(\alpha-r)t+\sigma\widetilde{W}_t] \\
&=S_0\exp\qty[\qty(\vc{r}-\delta-\frac{\sigma^2}{2})t+\sigma\vc{\widetilde{W}_t}].
\end{align*}
The final expression is very similar to the formula in \labelcref{it:bs-model-formulation}.
The only changes are that \(\alpha\) becomes \(r\), and \(W_t\) becomes
\(\widetilde{W}_t\).

\item It turns out that the stochastic process \(\{\widetilde{W_t}\}\) is a Weiner
process in the \emph{risk-neutral} world (not real world \warn{}). In the
risk-neutral world, the \emph{risk-neutral probability measure} \(\Q\) is used
for calculating probabilities, instead of the real-world probability measure
\(\mathbb{P}\). To justify that such risk-neutral world we ``imagine'' here
actually \emph{exists}\footnote{We can ``imagine'' many crazy things that do not
exist at all...}, many mathematical technicalities are involved, and they are
to be discussed in STAT3911. Here, we shall focus on the application of
risk-neutral pricing rather than its proof.

\item Consequently, in the risk-neutral world, the stock price process
\(\{S_t\}\) is still a geometric Brownian motion like the real world, with just
one change: The continuously compounded expected rate of return changes from
\(\alpha\) to the risk-free rate \(r\). So, the risky stock also earns
risk-free rate in this world, confirming that we are actually in the
\emph{risk-neutral} world.

Particularly, the distributions of stock price ratios and individual stock
prices in the risk-neutral world, where the probability measure \(\Q\) is used,
are given by
\[
\frac{S_{s+t}}{S_s}\overset{\mathrm{RN}}{\sim}\lndist{\qty(\vc{r}-\delta-\frac{\sigma^2}{2})t}{\sigma^{2}t}
\]
and
\[
S_t\overset{\mathrm{RN}}{\sim}\lndist{\ln S_0+\qty(\vc{r}-\delta-\frac{\sigma^2}{2})t}{\sigma^{2}t},
\]
for any \(s\ge 0\) and \(t\ge 0\).

\item \label{it:bs-rn-pricing-fmla}
Since every asset earns the risk-free rate \(r\) in the risk-neutral
world, we can calculate the price of a European
option/derivative\footnote{Pricing of American option/derivative under the
Black-Scholes model is beyond the scope.} using the risk-neutral pricing
formula, just like the case for binomial option pricing model: The current
option price \(V_0\) is given by
\[
V_0=\boxed{e^{-rT}\expvq{\text{terminal payoff}}}.
\]
Although the form of this formula is the same as that in the binomial tree
model, due to the great difference between the binomial tree model and
Black-Scholes model, the computation of \(\expvq{\text{terminal payoff}}\)
would be significantly different.

\item To illustrate the usage of risk-neutral pricing formula under
Black-Scholes model, we consider an example about pricing a \emph{power
derivative} on the stock \faIcon{apple-alt} (under Black-Scholes model), whose
payoff is \(V_T=S_T^{a}\) at the expiration time \(T\), where \(a\) is a fixed
real number.

Applying the risk-neutral pricing formula in \labelcref{it:bs-rn-pricing-fmla},
the current price of the power derivative is
\begin{align*}
V_0&=e^{-rT}\expvq{S_T^{a}} \\
&=e^{-rT}\expvq{\qty[S_0\exp(\qty(r-\delta-\frac{1}{2}\sigma^2)T+\sigma\widetilde{W}_T)]^{a}} \\
&=e^{-rT}\expvq{S_0^{a}\exp(a\qty(r-\delta-\frac{1}{2}\sigma^2)T+a\sigma\widetilde{W}_T)} \\
&=e^{-rT}S_0^{a}\exp[a\qty(r-\delta-\frac{1}{2}\sigma^2)T]M_{\widetilde{W}_T}(a\sigma)&\text{(\(M_{\widetilde{W}_T}\)
denotes the mgf of \(\widetilde{W}_T\), under the risk-neutral world)}.
\end{align*}
Since \(\widetilde{W}_T\overset{\mathrm{RN}}{\sim}\ndist{0}{\orc{T}}\), we have
\[
M_{\widetilde{W}_T}(\blc{a\sigma})
=\exp(\frac{\orc{T}(\blc{a\sigma})^{2}}{2}).
\]
It follows that
\[
V_0=e^{-rT}S_0^{a}\exp[a\qty(r-\delta-\frac{1}{2}\sigma^2)T]\exp(\frac{T(a\sigma)^{2}}{2})
=\boxed{S_0^{a}\exp{[(a-1)r-a\delta]T+\frac{1}{2}a(a-1)\sigma^2T}}.
\]
Special cases:
\begin{itemize}
\item \(a=0\): The time-\(T\) payoff of the power derivative is just
\(V_T=S_T^{0}=1\) always, so the power derivative is essentially a risk-free
bond with \(\$1\) payable at time \(T\). Using the formula derived here, the
current price is
\[
V_0=S_0^{0}\exp{[(0-1)r-0\delta]T+\frac{1}{2}0(0-1)\sigma^2T}
=e^{-rT},
\]
as expected.
\item \(a=1\): Then time-\(T\) payoff of the derivative is \(V_T=S_T\), the
time-\(T\) stock price. In other words, the derivative delivers one unit of
stock \faIcon{apple-alt} at time \(T\), and thus it can be treated as a prepaid
forward on the stock \faIcon{apple-alt}. Using the formula above, we get
\[
V_0=S_0^{1}\exp{[(1-1)r-1\delta]T+\frac{1}{2}1(1-1)\sigma^2T}
=S_0e^{-\delta T},
\]
which is precisely the current price of the \(T\)-year prepaid forward on the
stock.
\end{itemize}
\end{enumerate}
\subsection{The Black-Scholes Formula for European Call and Put}
\begin{enumerate}
\item By applying the risk-neutral pricing formula in
\labelcref{it:bs-rn-pricing-fmla} to price European call and put, we can obtain
the famous \emph{Black-Scholes formula}.

\item \label{it:bs-call-fmla}
 Consider a \(K\)-strike \(T\)-year European call on a stock \faIcon{apple-alt}
with dividends payable continuously at the dividend yield \(\delta\) under the
Black-Scholes model. Since the time-\(T\) payoff of the call is
\((S_T-K)_{+}\), using the risk-neutral pricing formula in
\labelcref{it:bs-rn-pricing-fmla} suggests that the current price of the call
is
\[
C_0=e^{-rT}\expvq{(S_T-K)_{+}}.
\]
After some tedious algebra (see STAT3905 for more details), we get
\[
C_0=\boxed{S_0e^{-\delta T}\Phi(d_1)-Ke^{-rT}\Phi(d_2)}
\]
where \(\displaystyle
d_1=\frac{\ln(S_0/K)+(r-\delta\blc{+}\frac{1}{2}\sigma^2)T}{\sigma\sqrt{T}}\),
\(\displaystyle
d_2=d_1-\sigma\sqrt{T}=\frac{\ln(S_0/K)+(r-\delta\blc{-}\frac{1}{2}\sigma^2)T}{\sigma\sqrt{T}}\),
and \(\Phi(\cdot)\) denotes the standard normal cdf.

\item \label{it:bs-pricing-fun}
We can express the formula in \labelcref{it:bs-call-fmla} using a generic
Black-Scholes type pricing function defined by:
\[
\bs{s_1,\delta_1}{s_2,\delta_2}{\sigma,T}=s_1e^{-\delta_1 T}\Phi(d_1)-s_2e^{-\delta_2 T}\Phi(d_2)
\]
where \(\displaystyle
d_1=\frac{\ln(s_1/s_2)+(\delta_2-\delta_1\blc{+}\frac{1}{2}\sigma^2)T}{\sigma\sqrt{T}}\)
and \(\displaystyle d_2=d_1-\sigma\sqrt{T}\). The notations carry the following
meanings:
\begin{itemize}
\item \(s_1\) is the current price of the asset \gc{gained} if the option is
exercised, and \(\delta_1\) is the ``dividend yield'' (continuously compounded
``growth rate'') of the asset.
\item \(s_2\) is the current price of the asset \rc{lost} if the option is
exercised, and \(\delta_2\) is the ``dividend yield'' of the asset.
\item \(\sigma\) is the annualized volatility of the asset whose prices follow
a geometric Brownian motion.
\item \(T\) is the time to maturity of the option.
\end{itemize}
By plugging in suitable arguments into the generic ``\(\operatorname{BS}\)''
function as suggested by the meanings above, it can be shown that
\(\bs{s_1,\delta_1}{s_2,\delta_2}{\sigma,T}\) gives the correct time-0 option
price. (See \Cref{subsect:exchange-opt} for more details.)

\begin{note}
The time-\(t\) option price (\(0<t<T\)) is given by
\(\bs{s_1,\delta_1}{s_2,\delta_2}{\sigma,T-t}\), by noting that the time to
maturity becomes \(T-t\), at time \(t\). Furthermore, \(s_1\) and \(s_2\) become the time-\(t\) prices.
\end{note}

\item Example: For the case of the call option, when it is exercised, the stock
\faIcon{apple-alt} is \gc{gained} and the ``pile of cash''
\faIcon{money-bill-wave} that worth the strike price is \rc{lost}. So:
\begin{itemize}
\item \(s_1\) should be the current price of the stock \faIcon{apple-alt}
(\(S_0\)), and \(\delta_1\) should be the dividend yield of \faIcon{apple-alt}
(\(\delta\)).
\item \(s_2\) should be the current price of the ``pile of cash''
\faIcon{money-bill-wave} (\(K\)), and \(\delta_2\) should be the ``dividend
yield'' of \faIcon{money-bill-wave}: the continuously compounded ``growth
rate'' of \faIcon{money-bill-wave}, namely the risk-free rate \(r\).
\item \(\sigma\) is the annualized volatility of the stock \faIcon{apple-alt}.
\end{itemize}
Hence, we have
\[
C_0=\boxed{\bs{S_0,\delta}{K,r}{\sigma,T}},
\]
which coincides with the formula in \labelcref{it:bs-call-fmla}.

\item For a \(K\)-strike \(T\)-year European put on \faIcon{apple-alt}, we can
again apply the generic Black-Scholes pricing function to get its time-0 price:
\[
P_0=\boxed{\bs{K,r}{S_0,\delta}{\sigma,T}}
=Ke^{-rT}\Phi(d_1^{P})-S_0e^{-\delta T}\Phi(d_2^{P}).
\]
Here we add a superscript ``\(P\)'' to each of \(d_1\) and \(d_2\) to emphasize
that they are for the European put, and are \emph{different} from the \(d_1^C\)
and \(d_2^C\) for the European call \warn{} (we also add the superscript \(C\)
here for clarity). More specifically, they are given by
\(\displaystyle
d_1^{P}=\frac{\ln(K/S_0)+(\delta-r+\frac{1}{2}\sigma^2)T}{\sigma\sqrt{T}}\) and
\(d_2^{P}=d_1^{P}-\sigma\sqrt{T}\).

In STAT3905, the formula for the time-0 put price is given by
\[
P_0=Ke^{-rT}\Phi(-d_2^{C})-S_0e^{-\delta T}\Phi(-d_1^{C}),
\]
which looks different from the expression above. However, we can show that
\(-d_2^C=d_1^P\) and \(-d_1^C=d_2^P\), so these two expressions are actually
equivalent.

\begin{pf}
Note that
\[
d_1^{P}=\frac{\ln(K/S_0)+(\delta-r\blc{+}\frac{1}{2}\sigma^2)T}{\sigma\sqrt{T}}
=-\frac{\ln(S_0/K)+(r-\delta\rc{-}\frac{1}{2}\sigma^2)T}{\sigma\sqrt{T}}
=-d_2^C,
\]
and
\[
d_2^{P}=\frac{\ln(K/S_0)+(\delta-r\rc{-}\frac{1}{2}\sigma^2)T}{\sigma\sqrt{T}}
=-\frac{\ln(S_0/K)+(r-\delta\blc{+}\frac{1}{2}\sigma^2)T}{\sigma\sqrt{T}}
=-d_1^C.
\]
\end{pf}
\item The Black-Scholes formula is not just limited to European call/put on
stocks paying dividends continuously. We can also use it to price European
option on some other kind of asset.

\item The first kind of asset we consider here is \emph{currency}
\faIcon{euro-sign}. We will use the Black-Scholes formula to price European
option on currency \faIcon{euro-sign}, which are also known as \defn{currency
option}.  The key idea is to treat currency as a ``special stock''. For a
currency option, there are generally \emph{two} currencies involved:
\begin{itemize}
\item \defn{underlying currency}: the currency on which the
option is written; the underlying asset for the option
\item \defn{benchmark currency}: the currency on which value is
measured/denominated
\end{itemize}
Suppose that the underlying currency is euro \faIcon{euro-sign} and the
benchmark currency is dollar \faIcon{dollar-sign}. Let \(X_t\) denotes the
time-\(t\) dollar/euro exchange rate, i.e., 1 euro can be exchanged for \(X_t\)
dollars at time \(t\), vice versa. Symbolically, \(\EUR{1}=\$ X_t\).

\item To price a \faIcon{dollar-sign}-denominated \(T\)-year \(\$K\)-strike European
option on \EUR{1} under the Black-Scholes model, we assume that the exchange
rate process \(\{X_t\}_{t\ge 0}\) is a geometric Brownian motion with constant
volatility \(\sigma\) and ``dividend yield''
\(r_{\mathrm{underlying}}=r_{\euro}\) (risk-free rate for the underlying
currency: euro). Also, the ``risk-free rate \(r\)'' in this case is the
risk-free rate for the \emph{benchmark} currency (dollar), denoted by
\(r_{\mathrm{benchmark}}=r_{\$}\). Thus, considering the risk-neutral world, we
can write
\[
X_t=X_0\exp[\qty(r_{\mathrm{benchmark}}-r_{\mathrm{underlying}}-\frac{\sigma^2}{2})t+\sigma\widetilde{W}_t]
\]
for any \(t\ge 0\). From this, we can then associate the notations here with
the ones for stock option as follows:
\begin{center}
\begin{tabular}{cc}
\toprule
Currency option&Stock option\\
\midrule
\(X_t\)&\(S_t\)\\
\(r_{\mathrm{benchmark}}\)&\(r\)\\
\(r_{\mathrm{underlying}}\)&\(\delta\)\\
\(\sigma\) (for exchange rates) &\(\sigma\) (for stock prices)\\
\bottomrule
\end{tabular}
\end{center}

\item Hence, the current prices of \faIcon{dollar-sign}-denominated \(T\)-year
\(\$K\)-strike European call and put on \EUR{1} are respectively
\[
C_0=\boxed{\bs{X_0,r_{\mathrm{underlying}}}{K,r_{\mathrm{benchmark}}}{\sigma,T}}
\]
and
\[
P_0=\boxed{\bs{K,r_{\mathrm{benchmark}}}{X_0,r_{\mathrm{underlying}}}{\sigma,T}}.
\]
\item The second kind of asset to be considered is \emph{stock paying discrete
dividends}. The underlying stocks we have dealt with so far pay continuous
proportional dividends at a dividend yield \(\delta\). What if the stock pays
discrete dividends instead? Here we shall consider the case where both the
amount and timing of the discrete dividend payments are known.

\item Consider a stock that pays dividend \(D_{t_i}\) at time \(t_i\), for
every \(i=1,\dotsc,n\), where \(0<t_1<\dotsb<t_n<T\). The time \(T\) is the
time to maturity of a European option on such stock.

The following shows an example of stock price path for a stock with discrete
dividend payable at time 2:
\begin{center}
\begin{tikzpicture}
\begin{axis}[domain=0:5, axis y line=left, axis x line=middle, ymin=-0.1, ymax=1.8, title=Stock price path, xmax=5.5,
xlabel=\(t\), ylabel=\(S_t\)]
\addplot[blue, domain=0:2]{e^x*sin(1/(x-1))+1};
\addplot[blue, domain=2:5]{e^x*sin(1/(x-1))+0.7};
\draw[blue] (2,1.13) circle [radius=0.5mm];
\draw[blue, fill] (2,0.83) circle [radius=0.5mm];
\draw[pen colour=orange, very thick, decorate,decoration={calligraphic brace, amplitude=5pt, raise=0}] (2,1.13) -- (2,0.83)
node[midway, right=1.5mm]{\(D_2\)};
\end{axis}
\end{tikzpicture}
\end{center}
Due to the no-arbitrage principle, the stock price jumps downward immediately
at each dividend payment time, by the amount of the dividend at that time. This
makes the stock price path \emph{discontinuous}. If the stock price process was
a geometric Brownian motion, the continuity of the sample path for the
underlying Weiner process would imply the continuity of the stock price path as
well. Hence, this suggests that the stock price process \(\{S_t\}_{t\ge 0}\)
cannot possibly be a geometric Brownian motion. So how should we apply the
Black-Scholes model to price a \(T\)-year \(K\)-strike option written on such
stock?

\item The key idea is to consider the \emph{prepaid forward price} instead.
From STAT3905, the time-\(t\) price of a prepaid forward for one share of stock to be
delivered at time \(T\) (\(\ge t\)) is
\[
F_{t,T}^{P}(S)=S_t-\operatorname{PV}_{t}(\text{dividends})
=S_t-\sum_{i:t_i>t}^{}D_{t_i}e^{-r(t_i-t)}.
\]
Although the stock price \(S_{t_j}\) has a downward jump of size \(D_{t_j}\) at each dividend
payment time \(t_j\), the corresponding present value
\(\operatorname{PV}_{t_j}(\text{dividends})\) drops by the same amount. This
ensures that the prepaid forward price \(F_{t,T}^{P}(S)\) is a continuous
function of time \(t\). Therefore, we can assume that the prepaid forward price
process \(\{F_{t,T}^{P}(S)\}_{t\in[0,T]}\) is a geometric Brownian motion, and
then we can proceed the option pricing, utilizing this assumption.

\item Since there is no dividend payable after time \(T\), the time-\(T\)
prepaid forward price is the same as the time-\(T\) stock price, i.e.,
\(F_{T,T}^{P}(S)=S_T\). Then, by the risk-neutral pricing formula in
\labelcref{it:bs-rn-pricing-fmla}, the current price of the European call on
the stock is
\[
C_0=e^{-rT}\expvq{(\blc{S_T}-K)_{+}}
=e^{-rT}\expvq{(\blc{F_{T,T}^{P}(S)}-K)_{+}}.
\]
From this, we can see that such European call can also be viewed as the
otherwise identical European call on the \emph{prepaid forward} on the stock,
also maturing at time \(T\), for option pricing purpose.

\item \label{it:bs-fmla-stock-disc-div}
With the assumption that \(\{F_{t,T}^{P}(S)\}_{t\in[0,T]}\) is a
geometric Brownian motion, we can use the generic Black-Scholes pricing
function to price such European call, with the following arguments:
\begin{itemize}
\item \(s_1=F_{0,T}^{P}(S)\): the time-0 price of the asset gained by
exercising the call, namely the prepaid forward maturing at time \(T\).
\item \(\delta=0\): the prepaid forward itself \faIcon{scroll} does not pay any
dividend.
\item \(\sigma\): the volatility of the \emph{prepaid forward}. That is, we
have
\[
\vari{\ln F_{t,T}^{P}(S)}=\sigma^2 t
\]
for any \(0\le t\le T\).
\end{itemize}
The current price of the European call is thus
\[
C_0=\boxed{\bs{F_{0,T}^{P}(S),0}{K,r}{\sigma,T}}.
\]
Similarly, the current price of the otherwise identical European \emph{put} is
\[
P_0=\boxed{\bs{K,r}{F_{0,T}^{P}(S),0}{\sigma,T}}.
\]
\end{enumerate}
\subsection{Pricing Power and Asian Options}
\label{subsect:power-asian}
\begin{enumerate}
\item By ``tweaking'' the generic Black-Scholes pricing function suitably, we
are able to price some options with complex payoff structure in the
Black-Scholes model also. One approach is to treat the underlying asset as a
``special stock'', somewhat like what we did for currency options, but in a
more complicated manner. We shall consider two examples in
\Cref{subsect:power-asian}: (European) power option and (geometric average
price) Asian option.

\item The first example here is (European) \emph{power options}. For any fixed
\(a\in\R\), an European \defn{power call option} (\defn{power put option}) has
a payoff \((S_T^{a}-K)_{+}\) (\((K-S_T^{a})_{+}\)) at the expiration time
\(T\). When \(a=1\), it reduces to the plain vanilla European call/put option.

\item The treatments for power call option and power put option are analogous,
so we focus on a power call option here. By \labelcref{it:bs-rn-pricing-fmla},
the current price of the option is
\[
C_0^{\mathrm{power}}=e^{-rT}\expvq{(\vc{S_T^{a}}-K)_{+}}.
\]
To actually compute this option price, we try to ``match'' the expression with
something we are more familiar with. By \labelcref{it:bs-call-fmla}, we know
that
\begin{align*}
e^{-rT}\expvq{(S_T-K)_{+}}
&=e^{-rT}\expvq{\qty(\vc{S_0\exp\qty[\qty(r-\delta-\frac{\sigma^2}{2})T+\sigma\widetilde{W}_T]}-K)_{+}} \\
&=\bs{S_0,\delta}{K,r}{\sigma,T} \\
&=S_0e^{-\delta T}\Phi(d_1)-Ke^{-rT}\Phi(d_2).
\end{align*}
So the idea is to express \(\vc{S_T^a}\) in the form of
\(\vc{S_0\exp\qty[\qty(r-\delta-\frac{\sigma^2}{2})t+\sigma\widetilde{W}_t]}\),
with possibly different inputs \(S_0,r,\delta,\sigma\), so that we can apply
this familiar formula.

\item After some algebra, we can write
\[
S_T^{a}=S_0^{a}\exp\qty[\gc{a\qty(r-\delta-\frac{\sigma^2}{2})T+a\sigma\widetilde{W}_T}]
=S_0^{a}\exp\qty[\gc{\qty(r-\delta^{*}-\frac{(\sigma^{*})^2}{2})T+\sigma^{*}\widetilde{W}_T}]
\]
where \(\sigma^{*}=a\sigma\) and
\(\delta^*=r-a(r-\delta)-\frac{1}{2}a(a-1)\sigma^2\). Practically, instead of
memorizing the formulas of \(\sigma^*\) and \(\delta^*\) here, we can solve the
following equation
\[
\blc{a\qty(r-\delta-\frac{\sigma^2}{2})T}+\mgc{a\sigma}\widetilde{W}_T
=\blc{\qty(r-\delta^{*}-\frac{(\sigma^{*})^2}{2})T}+\mgc{\sigma^{*}}\widetilde{W}_T
\]
for the unknowns \(\sigma^*\) and \(\delta^*\) by comparing coefficients, i.e.,
solving the following system:
\[
\begin{cases}
\displaystyle a\qty(r-\delta-\frac{\sigma^2}{2})T&=\displaystyle \qty(r-\delta^{*}-\frac{(\sigma^{*})^2}{2})T,\\
a\sigma&=\sigma^{*}.
\end{cases}
\]
With the expression in terms of \(\sigma^*\) and \(\delta^*\), we can price the
power option as follows:
\[
C_0^{\mathrm{power}}=e^{-rT}\expvq{\qty(\vc{S_0^{a}\exp\qty[\qty(r-\delta^{*}-\frac{(\sigma^{*})^2}{2})T+\sigma^{*}\widetilde{W}_T]}-K)_{+}}
=\bs{\rc{S_0^a},\rc{\delta^*}}{K,r}{\rc{\sigma^*},T}.
\]

\item Next, we consider a geometric average price call option with payoff
\((G_T-K)_{+}\) where \(G_T=(\prod_{i=1}^{n}S_{ih})^{1/n}\) with \(T=nh\). The
approach for pricing such option is the same, but the actual procedure would be
more complicated.

Example: Consider a 3-year geometric average price call with payoff
\((G_3-K)_{+}\) with \(G_3=(S_1\times S_2\times S_3)^{1/3}\). Like the usual
way to deal with the geometric average as discussed before, we first write
\begin{align*}
G_3&=\qty(S_0^{3}e^{\qty(r-\delta-\frac{\sigma^2}{2})(1)+\sigma\widetilde{W}_1}
e^{\qty(r-\delta-\frac{\sigma^2}{2})(2)+\sigma\widetilde{W}_2}
e^{\qty(r-\delta-\frac{\sigma^2}{2})(3)+\sigma\widetilde{W}_3}
)^{1/3} \\
&=S_0\exp\qty[\qty(r-\delta-\frac{\sigma^2}{2})\frac{1+2+3}{3}+\frac{\sigma}{3}(\widetilde{W}_1+\widetilde{W}_2+\widetilde{W}_3)] \\
&=S_0\exp\qty[\qty(r-\delta-\frac{\sigma^2}{2})(2)+\frac{\sigma}{3}(\widetilde{W}_1+\widetilde{W}_2+\widetilde{W}_3)].
\end{align*}
Now, note that \(\widetilde{W}_1+\widetilde{W}_2+\widetilde{W}_3
=3\widetilde{W}_1+2(\widetilde{W}_2-\widetilde{W}_1)+(\widetilde{W}_3-\widetilde{W}_2)\).
By the independent increment property, we have
\(\widetilde{W}_1+\widetilde{W}_2+\widetilde{W}_3\rnsim\ndist{0}{1^2+2^2+3^2}\equiv\ndist{0}{14}\).

Then, the key idea is to note that, in the risk-neutral world,
\[
\widetilde{W}_1+\widetilde{W}_2+\widetilde{W}_3\eqd\sqrt{\frac{14}{3}}\widetilde{W}_3
\]
since both follow the \(\ndist{0}{14}\) distribution. In view of this, our next
step is to find \(\delta^{*}\) and \(\sigma^*\) such that
\begin{align*}
S_0\exp\qty[\qty(r-\delta-\frac{\sigma^2}{2})(2)+\frac{\sigma}{3}(\widetilde{W}_1+\widetilde{W}_2+\widetilde{W}_3)]
&\eqd 
S_0\exp\qty[\gc{\qty(r-\delta-\frac{\sigma^2}{2})(2)+\frac{\sigma}{3}\sqrt{\frac{14}{3}}\widetilde{W}_3}]\\
&= S_0\exp\qty[\gc{\qty(r-\delta^{*}-\frac{(\sigma^{*})^2}{2})(3)+\sigma^{*}\widetilde{W}_3}].
\end{align*}
Practically, we need to solve the following equation
\[
\blc{\qty(r-\delta-\frac{\sigma^2}{2})(2)}+\mgc{\frac{\sigma}{3}\sqrt{\frac{14}{3}}}\widetilde{W}_3
=\blc{\qty(r-\delta^{*}-\frac{(\sigma^{*})^2}{2})(3)}+\mgc{\sigma^{*}}\widetilde{W}_3
\]
for the unknowns \(\sigma^{*}\) and \(\delta^{*}\), by comparing coefficients,
i.e., solving the following system:

\[
\begin{cases}
\displaystyle \qty(r-\delta-\frac{\sigma^2}{2})(2)&=\displaystyle \qty(r-\delta^{*}-\frac{(\sigma^{*})^2}{2})(3),\\
\displaystyle \frac{\sigma}{3}\sqrt{\frac{14}{3}}&=\sigma^{*}.\\
\end{cases}
\]
Finally, with the expression in terms of \(\sigma^*\) and \(\delta^*\), we can
price the geometric average price call option as follows:
\begin{align*}
C_0^{\mathrm{geometric}}&=e^{-rT}\expvq{(G_3-K)_{+}}\\
&=e^{-rT}\expvq{\qty(S_0\exp\qty[\qty(r-\delta^{*}-\frac{(\sigma^{*})^2}{2})(3)+\sigma^{*}\widetilde{W}_3]-K)_{+}} \\
&=\bs{S_0,\rc{\delta^*}}{K,r}{\rc{\sigma^*},T}.
\end{align*}
\end{enumerate}
\subsection{Exchange Options}
\label{subsect:exchange-opt}
\begin{enumerate}
\item \emph{Exchange option} is a complex option with particular importance. It
unifies the concepts of ``call'' and ``put'' options together, and view them
just as special cases of ``exchanging'' something:
\begin{itemize}
\item \emph{call:} buying an asset at a fixed strike price
\faIcon{arrows-alt-h} ``exchanging'' the strike price \faIcon{money-bill-wave} for the asset
\item \emph{put:} selling an asset at a fixed strike price
\faIcon{arrows-alt-h} ``exchanging'' the asset for the strike price \faIcon{money-bill-wave}
\end{itemize}
Generally, an exchange option can incorporate exchange of \emph{any two}
assets. Both assets can be risky, unlike the plain vanilla call/put options!

\item Setting: We are in a perfect market with two assets. For \(i=1,2\):
\begin{itemize}
\item \(S_t^{(i)}\) denotes the time-\(t\) price of asset \(i\).
\item The ``dividend yield'' of asset \(i\) is \(\delta^{(i)}\).
\end{itemize}
Given two such assets, we consider a \(T\)-year European exchange option which
gives us the right to exchange asset 2 for asset 1, or more precisely, give up
one unit of asset 2 in return for one unit of asset 1, at time \(T\). The
time-\(T\) payoff of the exchange option is then \((S_T^{(1)}-S_T^{(2)})_{+}\).

\item \label{it:ex-opt-assum} To price such exchange option, since two risky assets are involved, we
need a more general assumption than the usual assumption in the Black-Scholes
model. We shall assume that the process of the \emph{ratio} of the two asset
prices \(\displaystyle \qty{\frac{S_t^{(1)}}{S_t^{(2)}}}\) is a geometric
Brownian motion with volatility parameter \(\sigma\). Here, \(\sigma\) is the
volatility of the ratio of asset prices, so
\(\displaystyle \vari{\ln\frac{S_t^{(1)}}{S_t^{(2)}}}=\sigma^2 t\).

\item With this general assumption, we can obtain the following pricing formula
in terms of ``BS'' function.

\begin{theorem}
\label{thm:11-exchange-opt-fmla}
Under the assumption in \labelcref{it:ex-opt-assum}, the time-0 price of the
\(T\)-year European exchange option with payoff \((S_T^{(1)}-S_T^{(2)})_{+}\)
is
\[
V_0=\bs{S_0^{(1)},\delta^{(1)}}{S_0^{(2)},\delta^{(2)}}{\sigma,T}
=S_0^{(1)}e^{-\delta^{(1)}T}\Phi(d_1)-S_0^{(2)}e^{-\delta^{(2)}T}\Phi(d_2)
\]
where
\[
d_1=\frac{\ln(S_0^{(1)}/S_0^{(2)})+(\delta^{(2)}-\delta^{(1)}+\sigma^2/2)T}{\sigma\sqrt{T}}\qqtext{and}
d_2=d_1-\sigma\sqrt{T}.
\]
\end{theorem}
\begin{pf}
Take asset 2 as the ``benchmark'' asset (just like the benchmark currency).
Using cash as benchmark, the time-\(T\) payoff of the exchange option is
\(\$(S_T^{(1)}-S_T^{(2)})_{+}\). On the other hand, using asset 2 as benchmark, the
time-\(T\) payoff is
\[
\frac{\$(S_T^{(1)}-S_T^{(2)})_{+}}{\$ S_T^{(2)}}=
\qty(\frac{S_T^{(1)}}{S_T^{(2)}}-1)_{+}\text{ units of asset 2},
\]
which is just the time-\(T\) payoff of a \(1\)-strike plain vanilla European
call option on asset 1, with all values and prices expressed in the units of
asset 2. (Note that the time-\(T\) price of asset 1 is \(S_T^{(1)}/S_T^{(2)}\)
units of asset 2.) Pricing this plain vanilla call using the Black-Scholes call
price formula (\labelcref{it:bs-call-fmla}), the time-0 price is
\[
\bs{\frac{S_0^{(1)}}{S_0^{(2)}},\delta^{(1)}}{1,\delta^{(2)}}{\sigma,T}
=\frac{S_0^{(1)}}{S_0^{(2)}}e^{-\delta^{(1)}T}\Phi(d_1)-(1)e^{-\delta^{(2)}T}\Phi(d_2)
\text{ units of asset 2},
\]
where
\[
d_1=\frac{\ln(\frac{S_0^{(1)}}{S_0^{(2)}}/1)+(\delta^{(2)}-\delta^{(1)}+\sigma^2/2)T}{\sigma\sqrt{T}}\qqtext{and}
d_2=d_1-\sigma\sqrt{T}
\]
(same as the ones given in the theorem). To convert this price to the price in
the units of cash, we can just multiply it by \(S_0^{(2)}\) to get:
\[
V_0=S_0^{(2)}\qty[\frac{S_0^{(1)}}{S_0^{(2)}}e^{-\delta^{(1)}T}\Phi(d_1)-(1)e^{-\delta^{(2)}T}\Phi(d_2)]
=S_0^{(1)}e^{-\delta^{(1)}T}\Phi(d_1)-S_0^{(2)}e^{-\delta^{(2)}T}\Phi(d_2).
\]
\end{pf}

\item \label{it:blended-vol-fmla}
Note that the assumption in \labelcref{it:ex-opt-assum} is only about the
\emph{ratio} of asset prices, but not the individual asset prices. It does not
assume that the two asset prices are geometric Brownian motions individually.

However, if this is the case, it can be shown that the ratio of the prices
would be a geometric Brownian motion as well (see STAT3911 for more details).
More specifically, we assume that \(\{S_t^{(1)}\}\) and \(\{S_t^{(2)}\}\) are
geometric Brownian motion with volatilities \(\sigma_1\) and \(\sigma_2\)
respectively, and that the correlation coefficient \(\displaystyle \corr{\ln
\frac{S_t^{(1)}}{S_0^{(1)}}}{\ln \frac{S_t^{(2)}}{S_0^{(2)}}}=\corr{\ln
S_t^{(1)}}{\ln S_t^{(2)}}\) is a constant \(\rho\) always.  Then, the
volatility of the resulting geometric Brownian motion for the ratio of prices
can be obtained by
\[
\sigma=\boxed{\sqrt{\sigma_1^{2}+\sigma_2^{2}\rc{-}2\rho\sigma_1\sigma_2}}.
\]
\begin{warning}
We have \(\rc{-}2\rho\sigma_1\sigma_2\) rather than \(+2\rho\sigma_1\sigma_2\)
in the expression.
\end{warning}

This is sometimes known as the \defn{blended volatility} as it ``blends'' the
individual volatilities \(\sigma_1\) and \(\sigma_2\) together.

\begin{pf}
Note that
\begin{align*}
\sigma^2&=\vari{\ln\frac{S_1^{(1)}}{S_1^{(2)}}} \\
&=\vari{\ln S_1^{(1)}-\ln S_1^{(2)}} \\
&=\vari{\ln S_1^{(1)}}+\vari{\ln S_1^{(2)}}-2\cov{\ln S_1^{(1)}}{\ln S_1^{(2)}} \\
&=\sigma_1^2+\sigma_2^2-2\rho\sigma_1\sigma_2.
\end{align*}
\end{pf}

\item So far we have only considered ``one-to-one'' exchange option where one
unit of asset is exchanged for one unit of another asset. In general, exchange
option can also incorporate exchanging unequal numbers of assets. Consider a
European exchange option which gives us the right to obtain \(c_1\) units of
asset 1 by giving up \(c_2\) units of asset 2 at time \(T\). The time-\(T\)
payoff would then become \((c_1S_T^{(1)}-c_2S_t^{(2)})_{+}\). What would be the
impacts on the pricing?

\item \label{it:gen-exchange-opt-fmla} It turns out that we only need to
update two inputs in the ``BS'' function to get the time-0 price:
\[
V_0=\boxed{\bs{c_1S_0^{(1)},\delta^{(1)}}{c_2S_0^{(2)},\delta^{(2)}}{\sigma,T}}.
\]
\begin{pf}
We can see this by viewing such exchange option as a special kind of
``one-to-one'' exchange option, where one ``bundle'' \faIcon{box-open}
(``bundle 1'') containing \(c_1\) units of asset 1 is to be exchanged for
another ``bundle'' (``bundle 2'') \faIcon{box-open} containing \(c_2\) units of
asset 2.

The current prices of bundles 1 and 2 are \(c_1S_0^{(1)}\) and \(c_2S_0^{(2)}\)
respectively, hence the corresponding inputs in the ``BS'' function. The
``dividend yields'' of the bundles remain the same as the underlying asset as
the number of assets available do not affect the growth rate. Next, consider
\[
\vari{\ln\frac{c_1S_1^{(1)}}{c_2S_1^{(2)}}}
=\vari{\ln\frac{c_1}{c_2}+\ln\frac{S_1^{(1)}}{S_1^{(2)}}}
=\vari{\ln\frac{S_1^{(1)}}{S_1^{(2)}}}
=\sigma^2,
\]
so the volatility remains the same as well.
\end{pf}

\item As an application, we can use the concepts of exchange options to price
some other exotic derivatives: maximum and minimum claims. The idea is to
express rewrite the payoff such that some terms match with the ones for
exchange options, thereby ``decomposing'' the claims.

\item \label{it:max-claim-fmlas} Consider first a \(T\)-year maximum contingent
claim whose time-\(T\) payoff is \(\max\qty{S_T^{(1)},S_T^{(2)}}\), the maximum
of the two asset prices.  Writing
\[
\max\{S_T^{(1)},S_T^{(2)}\}=S_T^{(1)}+\qty(S_T^{(2)}-S_T^{(1)})_{+}
=S_T^{(2)}+\qty(S_T^{(1)}-S_T^{(2)})_{+},
\]
we see that the maximum claim can be decomposed into: (i) a \(T\)-year prepaid
forward on \underline{asset 1} + a time-\(T\) exchange option for
\underline{exchanging asset 1 for asset 2}; or (ii) a \(T\)-year prepaid
forward on \underline{asset 2} + a time-\(T\) exchange option for
\underline{exchanging asset 2 for asset 1}. These decompositions induce the
following formulas for the time-0 price of maximum claim:
\begin{enumerate}[label={(\roman*)}]
\item \(V_0^{\mathrm{max}}=\boxed{S_0^{(1)}e^{-\delta^{(1)}T}+\bs{S_0^{(2)},\delta^{(2)}}{S_0^{(1)},\delta^{(1)}}{\sigma,T}}\).
\item \(V_0^{\mathrm{max}}=\boxed{S_0^{(2)}e^{-\delta^{(2)}T}+\bs{S_0^{(1)},\delta^{(1)}}{S_0^{(2)},\delta^{(2)}}{\sigma,T}}\).
\end{enumerate}
\begin{warning}
Be careful about the orders of the inputs in the ``BS'' functions!
\end{warning}

\item \label{it:min-max-claim-relate} Next, consider a \(T\)-year minimum
contingent claim whose time-\(T\) payoff is \(\min\qty{S_T^{(1)},S_T^{(2)}}\),
the minimum of the two asset prices. We can utilize our knowledge on how to
price the maximum claim for pricing this minimum claim. First write
\[
\min\qty{S_T^{(1)},S_T^{(2)}}=S_T^{(1)}+S_T^{(2)}-\max\qty{S_T^{(1)},S_T^{(2)}}
\]
(which should be somewhat familiar to students who have taken STAT3909...).
With this decomposition, the time-0 price of the minimum claim can be obtained
by
\[
V_0^{\mathrm{min}}=\boxed{S_0^{(1)}e^{-\delta^{(1)}T}+S_0^{(2)}e^{-\delta^{(2)}T}-V_0^{\mathrm{max}}},
\]
where \(V_0^{\mathrm{max}}\) is the time-0 price of the corresponding maximum
claim.

\item \label{it:min-claim-dir-fmlas} Alternatively, we can use a more direct
approach as follows. Write instead (i)
\begin{align*}
\min\qty{S_T^{(1)},S_T^{(2)}}&=S_T^{(1)}+\min\qty{S_T^{(2)}-S_T^{(1)},0} \\
&=S_T^{(1)}\vc{-}\min\qty{\vc{S_T^{(1)}-S_T^{(2)}},0} \\
&=S_T^{(1)}-\qty(S_T^{(1)}-S_T^{(2)})_{+},
\end{align*}
or (ii)
\begin{align*}
\min\qty{S_T^{(1)},S_T^{(2)}}&=S_T^{(2)}+\min\qty{S_T^{(1)}-S_T^{(2)},0} \\
&=S_T^{(2)}\vc{-}\min\qty{\vc{S_T^{(2)}-S_T^{(1)}},0} \\
&=S_T^{(2)}-\qty(S_T^{(2)}-S_T^{(1)})_{+}.
\end{align*}
Consequently, the time-0 price of the minimum claim is given by
\begin{enumerate}[label={(\roman*)}]
\item \(V_0^{\mathrm{min}}=\boxed{S_0^{(1)}e^{-\delta^{(1)}T}\rc{-}\bs{S_0^{(1)},\delta^{(1)}}{S_0^{(2)},\delta^{(2)}}{\sigma,T}}\).
\item \(V_0^{\mathrm{min}}=\boxed{S_0^{(2)}e^{-\delta^{(2)}T}\rc{-}\bs{S_0^{(2)},\delta^{(2)}}{S_0^{(1)},\delta^{(1)}}{\sigma,T}}\).
\end{enumerate}
\begin{warning}
Again, be careful about the orders of the inputs in the ``BS'' functions!
\end{warning}
\end{enumerate}
